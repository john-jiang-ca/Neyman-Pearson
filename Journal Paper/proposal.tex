\documentclass[12pt,a4paper]{article}
\usepackage{amssymb}
\usepackage{mathtools}
\usepackage{graphicx}
\usepackage{epstopdf}
\usepackage{dsfont}
\usepackage{bbm}
\usepackage{setspace}
\usepackage[top=1in, bottom=1in, left=1in, right=1in]{geometry}

\makeatletter
\newcommand{\rmnum}[1]{\romannumeral #1}
\newcommand{\Rmnum}[1]{\expandafter\@slowromancap\romannumeral #1@}
\makeatother

\author{An Jiang \\
School of ECSE,\\
Mcgill University,\\
ID:   260501472}
\title{Some Properties for ROC of Extended Neyman Pearson Testing}
\begin{document}
\begin{spacing}{2.0}
\maketitle
\section*{Abstract}
Neyman Pearson testing(NP for short) is firstly proposed by J.Neyman and E.S.Pearson in 1933 in \cite{neyman1933problem}. After that they went on their study on NP testing and published a serious of papers on that. In 1939, Abraham Wald summarized the papers of Neyman and Pearson in \cite{wald1939contributions}. Dantzig, G.B. and Wald, discussed the questions of existence of optimal value and the necessary and sufficient conditions for optimal value for NP testing in \cite{dantzig1951fundamental}. Lemma summarized all of the conclusions given by previous people in \cite{LehmannTest}.  
There are three main objects for this report:
\\(\rmnum{1}) Find out whether the ROC for extended NP is convex.
\\(\rmnum{2}) Find out the direction of the norm of the tangent hyper plane at any point on the ROC.
\\(\rmnum{3}) Find out a method  to solve the following problem using extended NP:
\[
  max \;\;\;\;\;P_d
  \]
  \[
  s.t. \;\;\;\;\;\;P_{f_i}\leq c_i \;\;(i=1,2,...,m)
  \]
  Here $P_d$ represents for the probability of detection, $P_{f_i}$ represents for the probability of false alarm of the $i$'s hypotheses. The problem above has advantage over extended NP in engineering because extended NP has following form\cite{LehmannTest}:
  \[
  max \;\;\;\;\;P_d
  \]
  \[
  s.t. \;\;\;\;\;\;P_{f_i}= c_i \;\;(i=1,2,...,m)
  \]
  which means it confines that the probability of false alarm must be equal to some value. Instead in (\rmnum{3}), it only confines the probability of false alarm is no larger than the given value.

  The method that will be used to study these properties includes Taylor expansion, Delta function \cite{hassani2009dirac} and some conclusions from measure \cite{halmmm}. One challenge comes from the form of  $P_d$ and $P_{f_i}$ in extended NP, which are given in the following:
  \[
  P_d = \int \limits_{\mathcal{S}} f_0(x) \mathrm{d} x\,,
  \]
  \[
   P_{f_i} = \int \limits_{\mathcal{S}} f_i(x) \mathrm{d} x\,.
   \]
   The set of $\mathcal{S}$ is defined as following:
   \[
    \mathcal{S} :  \left\{x \bigg| \sum_{i=1}^{i=m}k_i\frac{f_i(x)}{f_0(x)} < 1 \right\}\,.
   \]
   The above form of $P_d$ and $P_{f_i}$ imply that it's hard to find out a close form expression for $P_d$ with respect to $P_{f_i}$ and that make computation hard. 
   
   A complete answer for the first two questions above will be given in the report and detail proof will be presented. A new method based on extended NP for solving the (\rmnum{3}) will be given with  proof.

% This is for the reference.
\bibliographystyle{ieeetr}	% (uses file "plain.bst")
\bibliography{mybib}		% expects file "myrefs.bib"
\end{spacing}
\end{document} 