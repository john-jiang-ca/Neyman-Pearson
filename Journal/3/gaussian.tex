% The Gaussian Case
\subsection{MROC Surface  under Gaussian Hypotheses}
In the following two examples, we compute the M-ROC under Gaussian Hypotheses. 

\noindent \textbf{Example 1:}
Assume three hypotheses given as \eqref{2015jan29a2}. We goal is using MENP to detect hypothesis $H_0$ against $\bar{H}_0$ and get the M-ROC surface.  
To form the M-ROC surface, we first consider points belong to $M_0$.
From previous discussion, we can see when $(P_d, c_1, c_2) \in M_0$, there exists non-negative $\mathbf{k}$ such that by using decision rule \eqref{equ: decision rule},
we have 
\begin{equation}
\begin{split}
\label{1125c0}
&P_d = \int_{-\infty}^{\infty} u(f_0(x) - \sum_{j=1}^{2}k_jf_j(x)) f_0(x)\mathrm{d}x    \,, \\
&P_{f_i} = \int_{-\infty}^{\infty} u(f_0(x) - \sum_{j=1}^{2}k_jf_j(x)) f_i(x) \mathrm{d}x = c_i\;\;\;\;\;    i=1, 2\,.
\end{split}
\end{equation}
We use Matlab to compute the M-ROC for region $M_0$. The values of $k_1$ and $k_2$ range from $0$ to $100$ in steps of $0.01$. Substituting the value of $k_1$ and $k_2$ into \eqref{1125c0}, results in the corresponding $P_d$ $P_{f_1}$ and $P_{f_2}$.  The set $M_0$ is illustrated in Figure \ref{pic: surface for m0 gaussian}. Figure \ref{pic: contour for m0 gaussian} presents the projection of Figure \ref{pic: surface for m0 gaussian} on the $c_1, c_2$ plane.



In Figure \ref{pic: contour for m0 gaussian} $N_0$ is the projection of $M_0$ on the $c_1, c_2$ plane. Since $M_0$ is the set of points with $[c_1, c_2] \in \alpha^+$, $N_0$ is the set of points belonging to $\alpha^+$.
Define curve $L_1$ as the set of points such that 
\[
\{ (c_1, c_2) \in L_1 | (c_1, c_2) \in {N}_0 \;\;\text{and} \;\;(c_1, c_2+\epsilon)\notin {N}_0 \;\;\;\;\text{for any positive $\epsilon$} \}\,.
\]
Define curve $L_2$ as the set of points such that 

\[
\{ (c_1, c_2) \in L_2 | (c_1, c_2) \in {N}_0 \;\;\text{and} \;\;(c_1 + \epsilon, c_2)\notin {N}_0 \;\;\;\;\text{for any positive $\epsilon$} \}\,.
\]
Let $N_1$ denote the region enclosed by line $c_1 = 0$, $c_2$; line $c_1$, $c_2 = 1$ and curve $L_1$.
Let $N_2$ denote the region enclosed by line $c_1 = 1$, $c_2$; line $c_1$, $c_2 = 0$ and curve $L_2$.
The regions of $N_0$ $N_1$ and $N_2$ are shown in Figure \ref{pic: contour for m0 gaussian}.

In the following, we present the decision rule for points belong to region $N_1$ and $N_2$.

\noindent \textbf{Property 2:}
\textit{\\(1) All points belonging to region $N_1$ or curve $L_1$, if they have the same $c_1$, they have the same decision rule and same $P_d$.
\\(2) All points belonging to region $N_2$ or curve $L_2$, if they have the same $c_2$, they have the same decision rule and same $P_d$.
}

\noindent \textbf{PROOF}
Recall $F(\mathbf{c})$ is defined as the largest $P_d$ can be achieved under constraint $\mathbf{P}_f = \mathbf{c}$ and 
       $G(\mathbf{c})$ is defined as the largest $P_d$ can be achieved under constraint $\mathbf{P}_f \leq \mathbf{c}$.
Firstly we will show $F(\mathbf{c}) = G(\mathbf{c}) $ when $\mathbf{c} \in \alpha^+$.

According to the definition of $\alpha^+$, for a point $(c_1^0, c_2^0) \in \alpha^+$, there exists a decision rule $\delta$ in form of \eqref{2015mar24}  with non-negative $k_i$  
such that under decision rule $\delta$, we have 
$\mathbf{P}_{f}(\delta) = \mathbf{c}\,.$
From \textbf{ENP Lemma (\rmnum{2})}, we know 
$P_d(\delta) = F(\mathbf{c})\,$.
Since $k_1, k_2 \geq 0$, according to \textbf{ENP Lemma (\rmnum{3})}, $\delta$ also achieve the largest $P_d$ while keeping $\mathbf{P_f} \leq \mathbf{c}$, i.e. 
 $P_d(\delta) =G(\mathbf{c}) $
From above discussion  we can see $F(\mathbf{c}) = G(\mathbf{c})$ for $\mathbf{c} \in \alpha^+$.
In the proof of \textbf{Lemma 1},  we have shown $G(\mathbf{c})$ is a non-decreasing function with  $\mathbf{c}$, it can be concluded that $F(\mathbf{c})$ is a non-decreasing function for $\mathbf{c} \in \alpha^+$.

Now consider point A in region ${N}_1$ with coordinate $(c_1, c_2) = (c_{1_A}, c_{2_A})$ (As it is shown in Fig. \ref{pic: contour for m0 gaussian}). The decision rule for point A can be computed through MENP (\rmnum{2}). To do it,  we need to determine set $\mathcal{C}$, which is the intersection of $\mathcal{A}_c$ and $\alpha^+$. According to the definition of $\mathcal{A}_c$ and $\alpha^+$, we can see $\mathcal{C}$ is the set of points enclosed by curves $L_1$, $L_2$ and line $c_1 = c_{1_A}$, $c_2$. Then we need to find $\mathbf{c}^0 \in \mathcal{C}$ such that it maximum $F(\mathbf{c})$ among all $\mathbf{c} \in \mathcal{C}$. As it has been proved $F(\mathbf{c})$ is a   non-decreasing function for $\mathbf{c} \in \alpha^+$, $\mathbf{c}^0$ must have the largest components $c_1$ and $c_2$ among all $\mathbf{c} \in \mathcal{C}$. 

It can be observed from Figure \ref{pic: contour for m0 gaussian} point B ( with coordinate $(c_{1_B}, c_{2_B})$)has the largest $c_1$, $c_2$ components among all points belong to set $\mathcal{C}$. 
Let $\delta_B$ be the optical decision rule for point B, by optical solution we mean $\delta_B$ acquires the maximum $P_d$ while keeping $P_{f_1} \leq c_{1_B}$ $P_{f_2} \leq c_{2_B}$.
From \textbf{MENP (\rmnum{2})} we know  $\delta_B$ is also the optimal decision rule for point A, i.e. it maximum $P_d$ while keeping $P_{f_1} \leq c_{1_A}$ $P_{f_2} \leq c_{2_A}$. Since both points use the same decision rule, the probability of detection for both points are the same. Besides that, since point B belongs to $N_0$ and for any positive $\epsilon$ $(c_{1_B}, c_{2_B} + \epsilon)$ does not belong to $N_0$, we can conclude point B is on curve $L_1$. As point B has the largest $c_1$ component among all points belong to $\mathcal{C}$, we can see $c_{1B} = c_{1A}$. Hence we can conclude for two points respectively belong to $N_1$ and $L_1$, if they have the same $c_1$ component, they have the same decision rule and same $P_d$. 

Furthermore, we can see as long as A is in region $N_1$ its decision rule only depends on the value of $c_{1}$.  In other words, with $c_1 = c_{1_A}$ fixed,  when $c_{2}$ changes, the decision rule and $P_d$ remain the same. Hence we can conclude all points belonging to region $N_1$ or curve $L_1$, if they have the same $c_1$, they have the same decision rule and same $P_d$.

In the same way, we can prove: For points belong to region $N_2$ or curve $L_2$, if they have the same $c_2$ value, they have the same decision rule and same $P_d$.

Q.E.D

Since we have computed $P_d$ for $(c_1, c_2) \in N_0$ and curves $L_1$ and $L_2$ belong to $N_0$, we can get $P_d$ for $(c_1, c_2)$ belongs to $N_1$ and $N_2$ through \textbf{Property 2}. M-ROC surface for this example is given in Figure  \ref{pic: LJS}.

