\documentclass[10pt,a4paper]{report}
\usepackage{amssymb}
\usepackage{mathtools}
\usepackage{graphicx}
\usepackage{epstopdf}
\usepackage{dsfont}
\usepackage{bbm}


%For lemma, theorem, proposition, corllary, proof, definition,example and remark.
\newtheorem{theorem}{Theorem}[section]
\newtheorem{lemma}[theorem]{Lemma}
\newtheorem{proposition}[theorem]{Proposition}
\newtheorem{corollary}[theorem]{Corollary}

\newenvironment{proof}[1][Proof]{\begin{trivlist}
\item[\hskip \labelsep {\bfseries #1}]}{\end{trivlist}}
\newenvironment{definition}[1][Definition]{\begin{trivlist}
\item[\hskip \labelsep {\bfseries #1}]}{\end{trivlist}}
\newenvironment{example}[1][Example]{\begin{trivlist}
\item[\hskip \labelsep {\bfseries #1}]}{\end{trivlist}}
\newenvironment{remark}[1][Remark]{\begin{trivlist}
\item[\hskip \labelsep {\bfseries #1}]}{\end{trivlist}}

\newcommand{\qed}{\nobreak \ifvmode \relax \else
      \ifdim\lastskip<1.5em \hskip-\lastskip
      \hskip1.5em plus0em minus0.5em \fi \nobreak
      \vrule height0.75em width0.5em depth0.25em\fi}
\makeatletter
\newcommand{\rmnum}[1]{\romannumeral #1}
\newcommand{\Rmnum}[1]{\expandafter\@slowromancap\romannumeral #1@}
\makeatother
% End
\author{An Jiang}
\title{Some conclusions for the ROC of Neyman Pearson Testing}
\begin{document}
\maketitle
\section{Abstract of the Project}
Even though the properties of Neyman Pearson Testing have been studying for a long time, there are few research about the ROC properties of extended Neyman Pearson. In this report, some important properties for the ROC of extended Neyman Pearson will be studied, which will include: 1. The shape of ROC; 2. The direction of the norm for the tangent hyper plant; 3. The relationship between $P_d$ and the parameters in extended Neyman Pearson $k_i$. After that, a practical theorem for engineering derived from extended Neyman Pearson will be given. Matlab will be used to show some examples for the properties given above. 

\section{References plan to use}
Following, but not limited with the following, references will be used in the report.
\\ [1]Poor, H. Vincent. "An introduction to signal detection and estimation." New York, Springer-Verlag, 1988, 559 p. 1 (1988).
\\(2)Wald, Abraham. "Contributions to the theory of statistical estimation and testing hypotheses." The Annals of Mathematical Statistics (1939): 299-326.
\\(3)Dantzig, George B., and Abraham Wald. "On the fundamental lemma of Neyman and Pearson." The Annals of Mathematical Statistics 22.1 (1951): 87-93.
\\(4)Lehmann, Erich L., and Joseph P. Romano. Testing statistical hypotheses. Springer, 2005.
\\(5)Halmos, Paul R. "The range of a vector measure." Bull. Amer. Math. Soc 54.4 (1948): 416-421.
\\(6)Neyman, Jerzy, and Egon S. Pearson. "On the problem of the most efficient tests of statistical hypotheses." Philosophical Transactions of the Royal Society of London. Series A, Containing Papers of a Mathematical or Physical Character 231 (1933): 289-337.
\\(7)Hassani, Sadri. "Dirac delta function." Mathematical Methods (2009): 139-170.



\end{document} 