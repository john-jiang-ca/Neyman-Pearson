\section{Comparison Between the MENP Framework and Neyman Pearson Framework for Energy Detector with Perfect Side Information }

\subsection{System Model}
\def \CHISQUY[#1]{\frac{1}{#1 2^N\Gamma(N)}\left(\frac{y}{#1}\right)^{N-1}\exp\left(-\frac{y}{2#1}\right)}
We consider a cognitive radio system where the licensed frequency spectrum could be occupied by one of two distinct signals $\{s_A, s_B\}$ or it could be vacant. 
Assume the detector has perfect side information of primary users, i.e. for a specific time slot the detector knows which primary user could occupy the channel. For easy presentation, let $T_A$ presents the time slot during which the channel can be either free or occupied by primary signal $s_A$. Let $T_B$ denote the time slot during which the channel can be either free or occupied by primary signal $s_B$. We can see, during time slot $T_A$, primary user $s_B$ will not present in the channel and during time slot $T_B$ primary user $s_A$ will not present in the channel. Since the detector has perfect side information, it knows the beginnings and endings for $T_A$ and $T_B$. Let $H_0$ denote the hypothesis under which the channel is free,  $H_1$ denote the hypothesis under which the channel is occupied by $s_A$ and $H_2$ denote the hypothesis under which the channel is occupied by $s_B$, i.e.
\begin{equation}
\begin{cases}
H_0:\;\;\;\;\text{channel is free}\\
H_1:\;\;\;\;\text{channel is occupied by $s_A$}\\
H_2:\;\;\;\;\text{channel is occupied by $s_B$}\,.
\end{cases}
\end{equation} 

During time slot $T_A$, we  test $H_0$ against $H_1$; during time slot $T_B$, we test $H_0$ against $H_2$. 
Even though there are three hypotheses, for a specific time slot, it is still a two hypotheses testing problem.   
For both $T_A$ and $T_B$, we use an energy detector to decide the status of the channel. 
We use the same energy detection model and same signal model as it is given in section 4, thus we have
\begin{equation}
  \label{20150627a2}
  \begin{split}
    H_0:\;\;\;\;\;&f_0(y) = \CHISQUY[\sigma_0^2]\\
    H_1:\;\;\;\;\;&f_{1}(y) = \CHISQUY[\sigma_1^2]\\
    H_2:\;\;\;\;\;&f_{2}(y) = \CHISQUY[\sigma_2^2]\,.
  \end{split}
\end{equation}

In the following the decision rule during time slot $T_A$ is considered. From the definition, we know during $T_A$ only primary signal $s_A$ could present in the channel and we are interested to test $H_0$ against $H_1$. In this case, the Neyman Pearson decision rule can be written as
\begin{equation}
  \frac{f_1(y)}{f_0(y)} \substack{\bar{H}_0 \\ \geq \\ < \\ H_0} \tau\,.
\end{equation}
Substituting \eqref{20150627a2} into above equation leads to 
\begin{equation}
  \left(\frac{\sigma_0^2}{\sigma_1^2}\right)^N\exp\left( (\frac{1}{2\sigma_0^2} -  \frac{1}{2\sigma_1^2}  )y \right)\substack{\bar{H}_0 \\ \geq \\ < \\ H_0} \tau
\end{equation}
Defining 
\begin{equation}
  g(y) = \left(\frac{\sigma_0^2}{\sigma_1^2}\right)^N\exp\left( (\frac{1}{2\sigma_0^2} -  \frac{1}{2\sigma_1^2}  )y \right)
  \label{20150629a0}
\end{equation}
\eqref{20150629a0} can be written as
\begin{equation}
  g(y) \substack{\bar{H}_0 \\ \geq \\ < \\ H_0} \tau\,.
  \label{20150622a12}
\end{equation}
Since $(\frac{1}{2\sigma_0^2} -  \frac{1}{2\sigma_1^2}  ) >  0$, we know $g(y)$ is a monotonic increasing function and $g^{-1}(y) $ exists.  
Let $V_\tau = g^{-1}(\tau)$ and \eqref{20150622a12} can be written in the form 
\begin{equation}
  y  \substack{\bar{H}_0 \\ \geq \\ < \\ H_0} V_\tau\,.
  \label{20150622a22}
\end{equation}
By using decision rule \eqref{20150622a22}, the probability of detection and the probability of false alarm can be written in form of 
$  P_d = \int_{0}^{V_\tau} f_0(y) \mathrm{d}y = F_0(V_\tau)$ and
$  P_f = \int_{0}^{V_\tau} f_1(y) \mathrm{d}y= F_1(V_\tau)$.
In order to ensure $P_f \leq c$, we should choose the threshold $V_\tau$ such that $F_1(V_\tau) = c$ is satisfied, i.e. $V_\tau = F^{-1}_1(c)$. 

By using the same method as in deriving the Neyman Pearson decision rule for time slot $T_A$, we can easily get the Neyman Pearson decision rule for time slot $T_B$. The overall decision rule can be written as following:
\begin{equation}
  \delta_{NP}:\;\;
\begin{cases}
 y  \substack{\bar{H}_0 \\ \geq \\ < \\ H_0} V_\tau\;\;\;\;V_\tau = F_1^{-1}(c)\;\;\;\;\text{during time slot $T_A$}\\
y  \substack{\bar{H}_0 \\ \geq \\ < \\ H_0} V_\tau'\;\;\;\;V_\tau' = F_2^{-1}(c)\;\;\;\;\text{during time slot $T_B$}\,,
\end{cases}
\label{20150702b1}
\end{equation}
and 
\begin{equation}
  P_d(\delta_{NP}) = \begin{cases}
    F_0(V_\tau)\;\;\;\;&\text{during time slot $T_A$}\\
    F_0(V_\tau')\;\;\;\;&\text{during time slot $T_B$}\,,
  \end{cases}
  \label{20150702b2}
\end{equation} 
\begin{equation}
  P_f(\delta_{NP}) = c\;\;\;\;\text{during time slot $T_A$ or $T_B$}\,.\;\;\;\;
\end{equation}

\subsection{Performance Comparison}
In this section, we compare the performance of the NP test with the performance of the MENP test for both time slot $T_A$ and $T_B$. For easy presentation, let $\delta_{M}$ denote the MENP decision rule and $P_d(\delta_M)$ denote the probability of detection achieved by decision rule $\delta_{M}$.  
Let $\delta_{NP} $ denote the NP decision rule and $P_d(\delta_{NP})$ denote the probability of detection achieved by decision rule $\delta_{NP}$. 
Similar to section 4.1.2, we set $\sigma_n^2= 0.1$, $\sigma_{s_A}^2 = 0.05$ $\sigma_{s_B}^2 = 0.15$ and $N=20$. 
The performance of the NP test with perfect side information is illustrated in Figure \ref{pic:20150702a0}.  
During time slot $T_A$, only primary signal $s_A$ could be present in the channel. Under the MENP framework, $P_{f_1}$ is the probability that the detector recognizes that the channel is free while $s_A$ is present. 
Hence under the MENP framework, we consider the relationship between $P_d$ and $c_1$. The performance of the MENP test is illustrated in Figure \ref{pic:20150702a0}.  

As we can observe, with $c_2$  fixed to $0.001$ (or $0.0015$), when $c_1$ is smaller than $0.12$ (or $0.15$) the performance of the MENP test and the NP test are the same; when $c_1$ is larger than $0.12$ (or $0.15$) the NP test has better performance. This is reasonable.  
For both MENP and NP tests with perfect side information, the decision rule can be written in form of $  y  \substack{\bar{H}_0 \\ \geq \\ < \\ H_0} V_\tau $ and the probability of detection is $ P_d = F_0(V_\tau)$. This suggests $P_d$ is an increasing function with $V_\tau$. 
In the MENP test, $V_\tau$ is determined by $V_{\tau MENP}= \min (F_1^{-1}(c_1), F_2^{-1}(c_2))$; in the NP test with perfect side information, during time slot $T_A$, $V_\tau$ is determined by $V_{\tau NP A} = F_1^{-1}(c)$. When $c_1 \leq F_1(F_2^{-1}(c_2))$, i.e. $F_1^{-1}(c_1) \leq  F_2^{-1}(c_2)$, we have  $V_{\tau MENP} = F_1^{-1}(c_1)$. Since $V_{\tau NP A} = F_1^{-1}(c)$, when $c = c_1$, we have $V_{\tau MENP} = V_{\tau NPA}$, i.e.  $P_d(\delta_M) = P_d(\delta_{NP})$.    
When $c_1 > F_1(F_2^{-1}(c_2))$, i.e. $F_1^{-1}(c_1) > F_2^{-1}(c_2)$, we have  $V_{\tau MENP} = F_2^{-1}(c_2) < F_1^{-1}(c_1)$. Since $V_{\tau NP A} = F_1^{-1}(c)$, when $c = c_1$, we have 
$V_{\tau MENP} < V_{\tau NP}$, i.e.  $P_d(\delta_M) < P_d(\delta_{NP})$.
Hence for the MENP test, with $c_2$ fixed, when $c_1$ is smaller than $F_1( F_2^{-1}(c_2) ) $, $P_d(\delta_M) = P_d(\delta_{NP})$; when $c_1$ is larger than $ F_1( F_2^{-1}(c_2))$, $P_d(\delta_M) < P_d(\delta_{NP})$. 
However for the situation when  $c_2$ is fixed to $0.15$, for $c_1 \in [0, 0.2]$, $c_1 <  F_1(F_2^{-1}(c_2))$ is always holds. Hence for the situation when $c_2$ is fixed to $0.15$ and $c_1 = c \leq 0.2$, the performance of the MENP test and the NP test are the same. This is seen in Figure \ref{pic:20150702a0}. 

Next we consider the performance of the NP test and the MENP test during time slot $T_B$.
The performance of the NP and the MNEP tests are shown in Figure \ref{pic:20150704a0}. During time slot $T_B$, only primary user $s_B$ could present in the channel. Under the MENP framework, we consider the relationship between $P_d$ and $c_2$. 

By using the same method as in analyzing the performance for the NP test and MENP test for time slot $T_A$, we can conclude that
 for the MENP test, with $c_1$ fixed, when $c_2$ is smaller than $ F_2( F_1^{-1}(c_1)) $, $P_d(\delta_M) = P_d(\delta_{NP})$; when $c_2$ is larger than $ F_2( F_1^{-1}(c_1)) $, $P_d(\delta_M) < P_d(\delta_{NP})$. 

From the above discussion, we can see that the NP test outperforms the MENP test when the detector has perfect side information. With $c_1$ ($c_2$) fixed, when $c_2$ ($c_1$) is small the performance of the  MENP test and the NP test are the same; when $c_2$ ($c_1$) becomes larger the NP test has better performance.  