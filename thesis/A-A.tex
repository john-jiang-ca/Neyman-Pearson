\resetdatestamp

\chapter{Numerical Results Program Guide}
\label{A:LaTeXmacros}

%\section{}
This section explains how to use the Matlab code that was developed to generate the numerical results in this thesis.  The following table provides an overview of each source code file. All of these files can be found on the attached CD. 

\begin{table}[h]
\begin{tabular}{l|p{350pt}}
\hline
\hline
Source File Name                  & Description                                                                \\ \hline
gau\_roc.m      & Compute ROC for general Gaussian situation.              \\
QZ\_Algorithm.m & Implements algorithm proposed in \cite{zhang2000efficient} to get MENP parameters.                \\
gau\_equvar.m   & Compute MROC for Gaussian distributions with equal variances. \\
general\_gau.m         & Compute MROC for general Gaussian situations.                   \\
mychisquare.m            & Compute MROC for Chi-Square situation.                        \\
cycloPDFs.m              & Compute PDFs for cyclostationary detection.                 \\ 
general\_pdfs.m           & Compute MROC for given PDFs.                                 \\
cyclo.m				     & Compute MROC for cyclostationary detection.             \\		
plotfigure.m             & Plot ROC or MROC figure. \\
init.txt                   & Record the adjustable parameters for above programs.         \\
\hline
\end{tabular}
\label{filelist}
\caption{Matlab source files}
\end{table}

To run a program, all of the files in Table A.1 must be placed in the Matlab current work folder. These programs have been successfully run using Matlab 2013b and Matlab 2014a on Linux platform with 12 GB RAM.  
To generate ROC surface for a Gaussian case, run gau\_roc.m. This can be done by running the following command in Matlab command line.
\\\texttt{Matlab$>$ gau\_roc}

The file loads configurations from init.txt file. When the computation finishes file gau\_roc.mat stores the results and plotfigure.m excutes automatically to print the ROC surface on the screen.


To generate the M-ROC surface for general Gaussian case, run general\_gau.m. This can be done by running the following command in Matlab command line.
\\\texttt{Matlab$>$ genaral\_gau}

The file loads configurations from init.txt file. When the computation finishes file general\_gau.mat stores the results and plotfigure.m excutes automatically to print the MROC surface on the screen.


To generate the M-ROC surface for the Gaussian case with equal variance, run gau\_equvar.m. This can be done by running the following command in Matlab command line.
\\\texttt{Matlab$>$ gau\_equvar}

The file loads configurations from init.txt file. When the computation finishes the file gau\_equvar.mat stores the results and plotfigure.m excutes automatically to print the MROC surface on the screen.

To generate the M-ROC surface for Chi-Square situation, run mychisquare.m. This can be done by running the following command in Matlab command line.
\\\texttt{Matlab$>$ mychisquare}

The file loads configurations from init.txt file. When the computation finishes file mychisquare.mat stores the results and plotfigure.m excutes automatically to print the MROC surface on the screen.

To generate the M-ROC surface cyclostationary detection, run cyclo.m. This can be done by running the following command in Matlab command line.
\\\texttt{Matlab$>$ cyclo}

The file loads configurations from init.txt file. When the computation finishes file PDFs.mat stores the PDFs for each hypothesis and  cyclo.mat stores the MROC data.

To compute MENP parameters for specific false alarm constraints, run QZ\_Algorithm.m with the false alarm constraints as arguments. This can be done by running following command in Matlab command line.
\\\texttt{Matlab$>$ QZ\_Algorithm(c1, c2)}

After the computation, the program prints the MENP parameters on the screen. 

The adjustable parameters for ENP (MENP) test are recorded in init.txt file and listed in Table A.2. 
\begin{table}[h]
\begin{tabular}{l|p{350pt}} 
\hline
\hline
Constant Name & Description                                                                                           \\
\hline
STEP\_K       & Step of ENP parameters, should be positive                                                            \\
RANGE\_K      & Range of ENP parameters, should be positive                                                           \\
MU0		  &  Mean of hypothesis $H_0$.\\
MU1		  & Mean of hypothesis $H_1$.\\
MU2       & Mean of hypothesis $H_2$.\\
VAR0      &    Variance of hypothesis $H_0$.\\
VAR1		&	Variance of hypothesis $H_1$.\\
VAR2		&		Variance of hypothesis $H_2$.\\
CHI\_DEGREE   & Degree of freedom in Chi-Square Example. Should be a positive integer.                                \\
DA\_LEN       & The length of data of OFDM. Should be an positive integer.                      \\
CP\_LEN       & The length of CP of OFDM. Should be an positive integer and smaller than DA\_LEN. \\
FRAME\_NUM    & The number of received OFDM frames. Should be an positive integer.\\
\hline                                   
\end{tabular}
\label{constantlist}
\caption{Adjustable Constants}
\end{table}

