\resetdatestamp

\chapter{\LaTeX{}Macros}
\label{A:LaTeXmacros}

\section{Numerical Results Overview and Guide }
This section explains how to use the Matlab code that were used to generate the numerical results in this thesis. The capabilities are also discussed. The following table provides an overview of each source code file. All of these files can be found on the attached CD. 

\begin{table}[h]
\begin{tabular}{ll}
\hline
\hline
Source File Name                  & Description                                                                \\ \hline
gaussian\_example.m      & Generate ROC for normal Gaussian situation.              \\
Qian\_Zhang\_Algorithm.m & Implements Qian Zhang Algorithm to achieve MENP parameters.                \\
gaussian\_equal\_var.m   & Generate MROC for Gaussian distributions with equal variances. \\
gaussian\_situ.m         & Generate MROC for normal Gaussian situation.                   \\
JAchisquare.m            & Generate MROC for Chi-Square situation.                        \\
energy.m                 & Generate MROC for energy detection.                            \\
cyclodetection.m         & Generate MROC for cyclostationary detection.                 \\ 
\hline
\end{tabular}
\label{filelist}
\caption{Matlab source files}
\end{table}

To run a program, all of the files in Table \ref{filelist} in the Matlab current work folder. These programs have been successfully run using Matlab 2013b and Matlab 2014a on Linux platform with 12 GB RAM.  When execute JAchisquare.m, gaussiansitu.m or gaussianexample.m, the program loads configurations from init.txt file. When the computation finishes file output.mat stores the result and plotfigure.m execute automatically to generate associated figure. 

The adjustable parameters for ENP (MENP) test are listed in table \ref{constantlist}. 
\begin{table}[h]
\begin{tabular}{l|p{350pt}} 
\hline
\hline
Constant Name & Description                                                                                           \\
\hline
STEP\_K       & Step of ENP parameters, should be positive                                                            \\
RANGE\_K      & Range of ENP parameters, should be positive                                                           \\
RANGE\_K\_NEG & Should be negative.                                                                                   \\
CHI\_DEGREE   & Degree of freedom in Chi-Square Example. Should be a positive integer.                                \\
SNR1          & The SNR of signal of hypothesis 1 in term of db.                                                      \\
SNR2          & The SNR of signal of hypothesis 2 in term of db.                                                      \\
SYM\_NUM      & The number of received symbols in energy detector. Should be positive.                                \\
DA\_LEN       & The length of data structure of an OFDM symbol. Should be an positive.integer                         \\
CP\_LEN       & The length of CP structure of an OFDM symbol. Should be an positive integer and smaller than DA\_LEN. \\
FRAME\_NUM    & The number of received OFDM frames. Should be an positive integer.\\
\hline                                   
\end{tabular}
\label{constantlist}
\caption{Adjustable Constants}
\end{table}
