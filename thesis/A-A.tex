\resetdatestamp

\chapter{\LaTeX{}Macros}
\label{A:LaTeXmacros}

\section{Numerical Results Overview and Guide }
This section explains how to use the Matlab code that were used to generate the numerical results in this thesis. The capabilities are also discussed. The following table provides an overview of each source code file. All of these files can be found on the attached CD. 

\begin{table}[h]
\begin{tabular}{ll}
\hline
\hline
Source File Name                  & Description                                                                \\ \hline
gaussian\_example.m      & Generate ROC for normal Gaussian situation.              \\
Qian\_Zhang\_Algorithm.m & Implements Qian Zhang Algorithm to achieve MENP parameters.                \\
gaussian\_equal\_var.m   & Generate MROC for Gaussian distributions with equal variances. \\
gaussian\_situ.m         & Generate MROC for normal Gaussian situation.                   \\
JAchisquare.m            & Generate MROC for Chi-Square situation.                        \\
energy.m                 & Generate MROC for energy detection.                            \\
cyclodetection.m         & Generate MROC for cyclostationary detection.                  
\hline
\end{tabular}
\label{filelist}
\captain{Matlab source files}
\end{table}

To run a program, all of the files in Table \ref{filelist} in the Matlab current work folder. These programs have been successfully run using Matlab 2013b and Matlab 2014a on Linux platform with 12 GB RAM.  When execute JAchisquare.m, gaussiansitu.m or gaussianexample.m, the program loads configurations from init.txt file. When the computation finishes file output.mat stores the result and plotfigure.m execute automatically to generate associated figure. 

The adjustable parameters for ENP (MENP) test are listed in the following table. 
