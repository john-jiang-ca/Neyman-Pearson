% cyclostationary based energy detection
\section{Cyclostationary Detection for multiple Primary Users}
\subsection{System Model}

We consider a cognitive radio system where the spectrum can be occupied by exactly one of two distinct primary signals $\{s_1, s_2\}$ or it could be vacant. Let $H_0$ denote the situation under which the channel is free, $H_1$ denote the hypothesis under which the channel is occupied by signal $s_1$ and $H_2$ denote the hypothesis under which the channel is occupied by signals $s_2$. Both $s_1, s_2$ are OFDM signals with exactly the same data structure, i.e. same length of CP and symbol length. The block diagram of the system is illustrated in Figure.

The measuring device observes the noise version of the signals that could be present in the channel and output a suitable testing statistics. With this statistics, the testing device employs MENP to determine the status of the channel. The input of measuring device is
\begin{equation}
  {x}[i] = \begin{cases}
	{n}[i]\;\;\;\;\;\;&\text{when $H_0$ is true}\\
	{n}[i]+s_1[i]\;\;\;\;\;\;&\text{when $H_1$ is true}\\
	{n}[i]+s_2[i]\;\;\;\;\;\;&\text{when $H_2$ is true}\\
  \end{cases}
  \label{equ:1209a1}
\end{equation}
where $n = \{0, 1, \cdots, N-1\}$. Assume each OFDM frame consists of  a CP sequence of length $l_C$ followed by a data sequence of length $l_D$. In general case, the receiver is not synchronized to the transmitted signal (c.f. ), i.e. $s_i[0]$ is not at the beginning of an OFDM symbol. Let $\tau$ represents the synchronization mismatch. That is, when $\tau = 0$, $s_i[0]$ is the first symbol of an OFDM frame; when $\tau = l_C+l_D -1$, $s_i[n]$ is the last symbol of an OFDM frame. Let $N = K(l_C + l_D)$, we can see when $\tau=0$, the detector would observe $K$ complete  OFDM symbols (as it is shown in Figure); otherwise, the detector will observe $K-1$ complete OFDM frame and $2$ incomplete OFDM frame (as it is shown in Figure). Like in most related literature (c.f. ), we assume  $n[i]$ are i.i.d. CSCG with variance $2\sigma_n^2$, i.e. $n[i] \sim \mathcal{CN}(0, 2\sigma_n^2)$. 

In this case, the measuring device is a cyclostationary detector, the output is the  cyclic covariance of the observations:
\begin{equation}
  y = 
  \label{<++>}
\end{equation}<++>


 

\subsection{Numerical Results}

