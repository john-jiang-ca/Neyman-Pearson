% cyclostationary based energy detection
\section{Cyclostationary Detection for multiple Primary Users}
\subsection{System Model}

We consider a cognitive radio system where the spectrum can be occupied by exactly one of two distinct primary signals $\{s_A, s_B\}$ or it could be vacant. Let $H_0$ denote the situation under which the channel is free, $H_1$ denote the hypothesis under which the channel is occupied by signal $s_A$ and $H_2$ denote the hypothesis under which the channel is occupied by signals $s_B$. Both $s_A, s_B$ are OFDM signals with exactly the same data structure. The block diagram of the system is illustrated in Figure.

The measuring device observes $N$ samples of the noise version signals that could be present in the channel and output a suitable testing statistics. With this statistics, the testing device employs MENP to determine the status of the channel. The input of measuring device is
\begin{equation}
  \mathbf{x} = \begin{cases}
	\mathbf{n}\;\;\;\;\;\;&\text{when $H_0$ is true}\\
	\mathbf{n}+\mathbf{s}_A\;\;\;\;\;\;&\text{when $H_1$ is true}\\
	\mathbf{n}+\mathbf{s}_B\;\;\;\;\;\;&\text{when $H_2$ is true}\\
  \end{cases}
  \label{equ:1209a1}
\end{equation}
where 
\begin{equation}
  \begin{cases}
	&\mathbf{x} = (x_0, x_1, \cdots, x_{N_1})\\
	&\mathbf{s}_A = (s_{A0}, s_{A1}, \cdots, s_{A(N-1)})\\
	&\mathbf{s}_B = (s_{A0}, s_{B1}, \cdots, s_{B(N-1)})\\
	&\mathbf{n} = (n_{0}, n_{1}, \cdots, n_{N-1})\,.
  \end{cases}
  \label{xssn}
\end{equation}
Assume each OFDM frame contains a CP sequence of length $l_C$ followed by a data sequence of length $l_D$, so the total length of an OFDM frame is $l_0 = l_C+l_D$. In general case, when there are signal transmitting, the receiver is not synchronized to the transmitted signal (c.f. ), i.e. $s_{A0}$ ($s_{B0}$) is not the first symbol of an OFDM frame. Let $\tau$ represents the synchronization mismatch. That is, when $\tau = 0$, $s_0$ is the first symbol of an OFDM frame; when $\tau = l_C+l_D -1$, $s_0$ is the last symbol of an OFDM frame. Let $N = Kl_0$, we can see when $\tau=0$, the detector would observe $K$ complete  OFDM frames (as it is shown in Figure); otherwise, the detector would observe $K-1$ complete OFDM frame and $2$ incomplete OFDM frame (as it is shown in Figure). Like in most related literature (c.f. ), we assume  $n_i$ are i.i.d. CSCG with variance $2\sigma_n^2$, i.e. $n_i \sim \mathcal{CN}(0, 2\sigma_n^2)$. 

Let $r_i=x_ix_j^\ast$ ($j=i+l_D$), (c.f. ) shows $r_i$ exhibit cyclic property at frequency $\alpha = \frac{K}{l_0}$ ($K = \pm1, \pm2, \cdots$). In this case, we consider a single cyclic detector at frequency $\alpha = \frac{1}{l_0}$. The output of the measuring device is 
\begin{equation}
  Y = \frac{1}{N}\sum_{i=0}^{N-1} r_i\exp(-j2\pi\alpha i)\,.
  \label{cyclic_cov}
\end{equation}
Let $a_i = \exp(-j2\pi\alpha i)$, then \eqref{cyclic_cov} can be written as
\begin{equation}
  Y = \frac{1}{N}\sum_{i=0}^{N-1} r_ia_i\,.
  \label{cyclic_cov_2}
\end{equation}
The testing device determines the status of the channel by observing the real and imaginary part of Y
By observing $y$, a realization  of $Y$, the testing device determines the status of the channel.

Let $\bar{c}$ and $\tilde{c}$ denote the real and imaginary parts of complex number $c$ respectively. Then, $a_i = \bar{a}_i + j\tilde{a}_i$, $y = \bar{y}+j\tilde{y}$  and $r_i = \bar{r}_i + j\tilde{r}_i$. 
In (c.f. ), the author shows $Y$ approaches normal distribution when $K$ goes to infinity. Thus we can employ a normal distribution for a large $K$. In such case, the vector $[\bar{y}, \tilde{y}]$ is governed by multivariate normal distribution. 

The moment function of $\bar{r}_i, \tilde{r}_i$ are derived in (c.f. ) and summarized in Table \ref{table1}.
\begin{table}[h]
  \label{table1}
  \begin{tabular}{|c|c|c|c|c|c|}
	\hline
    \multirow{2}{*}{}           & \multirow{2}{*}{$H_0$} & \multicolumn{2}{c|}{$H_1$}                                                               & \multicolumn{2}{c|}{$H_2$}                                                               \\ \cline{3-6} 
	&                        & situ1                                                   & situ2                          & situ1                                                   & situ2                          \\ \hline
	$E[\bar{r}_i]$              & $0$                    & $2\simga_{s_A}^2$                                       & $0$                            &                                                         & $0$                            \\ \hline
	$E[\tilde{r}_i]$            & $0$                    & $0$                                                     & $0$                            & $0$                                                     & $0$                            \\ \hline
	$E[\bar{r}_i^2]$            & $2\sigma_n^4$          & $8\simga_{s_A}^4+4\sigma_{s_A}^2\sigma_n^2+2\sigma_n^4$ & $2(\sigma_n^2+\sigma_{s_A}^2)$ & $8\simga_{s_B}^4+4\sigma_{s_B}^2\sigma_n^2+2\sigma_n^4$ & $2(\sigma_n^2+\sigma_{s_B}^2)$ \\ \hline
	$E[\tilde{r}_i^2]$          & $2\sigma_n^4$          & $4\sigma_{s_A}^2\sigma_n^2+2\sigma_n^4$                 & $2(\sigma_n^2+\sigma_{s_A}^2)$ & $4\sigma_{s_B}^2\sigma_n^2+2\sigma_n^4$                 & $2(\sigma_n^2+\sigma_{s_B}^2)$ \\ \hline
	$E[\bar{r}_i\bar{r}_j]$     & $0$                    & $0$                                                     & $0$                            & $0$                                                     & $0$                            \\ \hline
	$E[\bar{r}_i\tilde{r}_j]$   & $0$                    & $0$                                                     & $0$                            & $0$                                                     & $0$                            \\ \hline
	$E[\tilde{r}_i\tilde{r}_j]$ & $0$                    & $0$                                                     & $0$                            & $0$                                                     & $0$                            \\ \hline
  \end{tabular}
\end{table}


In the following, we consider the distribution of $[\bar{y}, \tilde{y}]$ under hypothesis $H_1$ with synchronization mismatch $\tau = \tau_0$.
From \eqref{cyclic_cov_2}, we have
\begin{equation}
  \begin{split}
  \bar{y} = &\Re{(\sum_{i=0}^{N-1} a_ir_i)}\\
  = &\Re(\sum_{i=0}^{N-1}(\bar{a}_i+j\tilde{a}_i)(\bar{r}_i+j\tilde{r}_i))\\
  = &\sum_{i=0}^{N-1}\bar{a}_i\bar{r}_i - \sum_{i=0}^{N-1}\tilde{a}_i\tilde{r}_i
\end{split}
  \label{R}
\end{equation}
and
\begin{equation}
  \begin{split}
    \tilde{y} = &\Im(\sum_{i=0}^{N-1} a_ir_i)\\
      = &\sum_{i=0}^{N-1}\tilde{a}_i\bar{r}_i - \sum_{i=0}^{N-1}\bar{a}_i\tilde{r}_i
  \end{split}
  \label{I}
\end{equation}

Let $\Theta_\tau$ denote the set of subscripts such that if $i \in \Theta_\tau$,  $s[i]$ belongs to a part of CP sequence.
From Figure XX, we can see $i \in \Theta_\tau$ if and only if $i + \tau \in [Ml_0, Ml_0+l_c -1]$ ($M = 0, 1, \cdots, K$). Inserting the statistics of $r_i$ from Table \ref{table1} into \eqref{R} and \eqref{I}, we have
\begin{equation}
\begin{split}
  E[\bar{y}] = &\sum_{i=0}^{N-1}\bar{a}_iE[\bar{r}_i] - \sum_{i=0}^{N-1}\tilde{a}_iE[\tilde{r}_i]\\
  = &\sum_{i\in\Theta_\tau}\bar{a}_i2\sigma_s^2
  \end{split}
  \label{ER}
\end{equation}
and
\begin{equation}
\begin{split}
  E[\tilde{y}] = &\sum_{i\in\Theta}\tilde{a}_i2\sigma_s^2\\
  = &\sum_{i\in\Theta}\tilde{a}_i2\sigma_s^2\,.
  \end{split}
  \label{EI}
\end{equation}

Hence we have 
\begin{equation}
\label{muy}
\mu_y = \sum_{i\in\Theta_\tau}\bar{a}_i2\sigma_s^2 + j\sum_{i\in\Theta}\tilde{a}_i2\sigma_s^2
\end{equation}

The seconde statistic of $\bar{y}$ and $\tilde{y}$ can be written as
\begin{equation}
\begin{split}
  E[\bar{y}^2] = &E[(\sum_{i=0}^{N-1} a_ir_i)(\sum_{i=0}^{N-1} a_ir_i)]\\
  = &E[\sum_{i=0}^{N-1}\sum_{k=0}^{N-1}\bar{a}_i\bar{a}_k\bar{r}_i\bar{r}_k - \sum_{i=0}^{N-1}\sum_{k=0}^{N-1}\bar{a}_i\tilde{a}_k\bar{r}_i\tilde{r}_k - \sum_{i=0}^{N-1}\sum_{k=0}^{N-1}\bar{a}_i\tilde{a}_k\bar{r}_i\tilde{r}_k + \sum_{i=0}^{N-1}\sum_{k=0}^{N-1}\tilde{a}_i\tilde{a}_k\tilde{r}_i\tilde{r}_k]
  \end{split}
  \label{ER}
\end{equation}

Submite the statistic of $\bar{r}$ and $\tilde{r}$, \eqref{ER} can be written as
\begin{equation}
  E[R^2] = \sum_{i=1}^N(\bar{a}_i^2E[\bar{r}^2] + \tilde{a}_i^2E[\tilde{r}_i^2])\,.
  \label{ER2}
\end{equation}

Using a similiar method, we have 
\begin{equation}
  E[I^2] = \sum_{i=1}^N(\bar{a}_i^2E[\tilde{r}^2] + \tilde{a}_i^2E[\bar{r}_i^2])
  \label{EI2}
\end{equation}
and
\begin{equation}
  E[RI]=\sum_{i=1}^N\tilde{a}_i\bar{a}_jE[\bar{r}_i^2] + \sum_{i=1}^N\tilde{a}_i\bar{a}_jE[\tilde{r}_i^2] \,.
  \label{ERI}
\end{equation}





\subsection{Numerical Results}

