% cyclostationary based energy detection
\section{Cyclostationary Detection for multiple Primary Users}
\subsection{System Model}

We consider a cognitive radio system where the spectrum can be occupied by exactly one of two distinct primary signals $\{s_A, s_B\}$ or it could be vacant. Let $H_0$ denote the situation under which the channel is free, $H_1$ denote the hypothesis under which the channel is occupied by signal $s_A$ and $H_2$ denote the hypothesis under which the channel is occupied by signals $s_B$. Both $s_A, s_B$ are OFDM signals with same frame structure. The block diagram of the system is illustrated in Figure.

The measuring device observes $M$ samples of the noise version signals that could be present in the channel and output a suitable testing statistics. With this statistics, the testing device employs MENP to determine the status of the channel. The input of measuring device is
\begin{equation}
  \mathbf{x} = \begin{cases}
	\mathbf{n}\;\;\;\;\;\;&\text{when $H_0$ is true}\\
	\mathbf{n}+\mathbf{s}_A\;\;\;\;\;\;&\text{when $H_1$ is true}\\
	\mathbf{n}+\mathbf{s}_B\;\;\;\;\;\;&\text{when $H_2$ is true}\\
  \end{cases}
  \label{equ:1209a1}
\end{equation}
where 
\begin{equation}
  \begin{cases}
	&\mathbf{x} = (x_0, x_1, \cdots, x_{M-1})\\
	&\mathbf{s}_A = (s_{A0}, s_{A1}, \cdots, s_{A(M-1)})\\
	&\mathbf{s}_B = (s_{B0}, s_{B1}, \cdots, s_{B(M-1)})\\
	&\mathbf{n} = (n_{0}, n_{1}, \cdots, n_{M-1})\,.
  \end{cases}
  \label{xssn}
\end{equation}
Assume each OFDM frame contains a CP sequence of length $l_C$ followed by a data sequence of length $l_D$, so the total length of an OFDM frame is $l_0 = l_C+l_D$. In general case, when there are signal transmitting, the receiver is not synchronized to the transmitted signal (c.f. ), i.e. $s_{A0}$ ( or $s_{B0}$) is not the first symbol of an OFDM frame. Let $\tau$ represents the synchronization mismatch. That is, when $\tau = 0$, $s_0$ is the first symbol of an OFDM frame; when $\tau = l_C+l_D -1$, $s_0$ is the last symbol of an OFDM frame. Let $M = Kl_0$, i.e. when perfect synchronization the detector would observe $K$ complete  OFDM frames (as it is shown in Figure); otherwise, the detector would observe $K-1$ complete OFDM frames and $2$ incomplete OFDM frames (as it is shown in Figure). We also assume  $n_i$ are i.i.d. CSCG with variance $2\sigma_n^2$, i.e. $n_i \sim \mathcal{CN}(0, 2\sigma_n^2)$. 

The CP structure could introduce strong cyclostationarity to the transmitted OFDM signals ($s_A$ or $s_B$) \cite{lunden2010robust}. 
Let $r_i=x_ix_j^\ast$ ($j=i+l_D$), \cite{lunden2007spectrum} shows $r_i$ exhibit cyclic property at frequency $\alpha = \frac{Q}{l_0}$ ($Q = \pm1, \pm2, \cdots$). In this case, we consider a single cyclic detector at frequency $\alpha = \frac{1}{l_0}$. 
Let $\mathbf{r}$ denote the vector of $r_i$, since the length of vector $\mathbf{x}$ is $M$, the length of $\mathbf{r}$ is $M - l_D$. For simple representation, let $N = M - l_D$.
Like in most related literature reviews (c.f. \cite{lunden2010robust} \cite{dandawate1994statistical}), the cyclic-covariance estimator for frequency $\alpha$ can be written as
\begin{equation}
  \hat{R}(l_D, \alpha) = \frac{1}{N}\sum_{i=0}^{N-1} r_ia_i\,.
  \label{cyclicR}
\end{equation}
where $a_i = \exp(-j2\pi\alpha i)$. 
The output of the measuring device is the real and imaginary part of the cyclic covariance estimator, i.e. 
\begin{equation}
  Y = \begin{bmatrix}
	R \\
	I
  \end{bmatrix}\,,
  \label{cyclic_cov}
\end{equation}
where 
\[
  R = \Re(\hat{R}(l_D, \alpha))
\]
and 
\[
  I = \Im(\hat{R}(l_D, \alpha))\,.
\]
By observing $y$, a realization  of $Y$, the testing device determines the status of the channel.
According to \cite{lunden2010robust}, $\hat{R}(l_D, \alpha)$ subject to normal distribution for a large $K$, so $Y$ has the two-dimensional Gaussian distribution \cite{goodman1963statistical}.
Let $\bar{c}$ and $\tilde{c}$ denote the real and imaginary parts of a complex number $c$ respectively. Then, $a_i = \bar{a}_i + j\tilde{a}_i$  and $r_i = \bar{r}_i + j\tilde{r}_i$. 

The moment function of $\bar{r}_i, \tilde{r}_i$ are derived in \cite{axell2011optimal} and summarized in Table \ref{table1} on page \pageref{table1}.

% Please add the following required packages to your document preamble:
% \usepackage{multirow}
\begin{table}[h]
  \begin{tabular}{|c|c|c|c|c|c|}
	\hline
	\multirow{2}{*}{}           & \multirow{2}{*}{$H_0$} & \multicolumn{2}{c|}{$H_1$}                                                               & \multicolumn{2}{c|}{$H_2$}                                                               \\ \cline{3-6} 
	&                        & situ1                                                   & situ2                          & situ1                                                   & situ2                          \\ \hline
	$E[\bar{r}_i]$              & $0$                    & $2\sigma_{s_A}^2$                                       & $0$                            & $2\sigma_{s_N}^2$                                                         & $0$                            \\ \hline
	$E[\tilde{r}_i]$            & $0$                    & $0$                                                     & $0$                            & $0$                                                     & $0$                            \\ \hline
	$E[\bar{r}_i^2]$            & $2\sigma_n^4$          & $8\sigma_{s_A}^4+4\sigma_{s_A}^2\sigma_n^2+2\sigma_n^4$ & $2(\sigma_n^2+\sigma_{s_A}^2)^2$ & $8\sigma_{s_B}^4+4\sigma_{s_B}^2\sigma_n^2+2\sigma_n^4$ & $2(\sigma_n^2+\sigma_{s_B}^2)^2$ \\ \hline
	$E[\tilde{r}_i^2]$          & $2\sigma_n^4$          & $4\sigma_{s_A}^2\sigma_n^2+2\sigma_n^4$                 & $2(\sigma_n^2+\sigma_{s_A}^2)^2$ & $4\sigma_{s_B}^2\sigma_n^2+2\sigma_n^4$                 & $2(\sigma_n^2+\sigma_{s_B}^2)^2$ \\ \hline
	$E[\bar{r}_i\tilde{r}_i]$   & $0$                    & $0$                                                     & $0$                            & $0$                                                     & $0$                            \\ \hline
  \end{tabular}
  \caption{Moment function of $r_i$}
  \label{table1}
\end{table}
Next we consider the statistic of $r_i$ and $r_k$ ($i < j$), for easy presentation, we let $j=i+l_D$ and $l=k+l_D$,
\begin{equation}
  \begin{split}
    E[\bar{r}_i\bar{r}_k]= 
  \end{split}<++>
  \label{<++>}
\end{equation}<++>

In the following, we consider the distribution of $\begin{bmatrix}
  R \\
  I
\end{bmatrix}$ under hypothesis $H_1$ with synchronization mismatch $\tau = \tau_0$.
From the definition of $R$ and $I$, we have 
\begin{equation}
  \begin{split}
	R = &\Re{(\sum_{i=0}^{N-1} a_ir_i)}\\
	= &\Re(\sum_{i=0}^{N-1}(\bar{a}_i+j\tilde{a}_i)(\bar{r}_i+j\tilde{r}_i))\\
	= &\sum_{i=0}^{N-1}\bar{a}_i\bar{r}_i - \sum_{i=0}^{N-1}\tilde{a}_i\tilde{r}_i
  \end{split}
  \label{R}
\end{equation}
and
\begin{equation}
  \begin{split}
	I = &\Im(\sum_{i=0}^{N-1} a_ir_i)\\
	= &\sum_{i=0}^{N-1}\tilde{a}_i\bar{r}_i + \sum_{i=0}^{N-1}\bar{a}_i\tilde{r}_i
  \end{split}
  \label{I}
\end{equation}

Let $\Theta_{\tau_0}$ denote the set of subscripts such that if $i \in \Theta_{\tau_0}$,  $s_i$ belongs to  CP sequence of an OFDM frame.
From Figure XX, we can see $i \in \Theta_{\tau_0}$ if and only if $i + \tau_0 \in [Ml_0, Ml_0+l_c -1]$ ($M = 0, 1, \cdots, K$). 

Let $\mu_{R|\tau_0}$ $\mu_{I|\tau_0}$ denote the mean of $R$ and $I$ with synchronization mismatch $\tau=\tau_0$ respectively. By using the statistics of $r_i$ in Table \ref{table1}, we compute  $\mu_{R|\tau_0}$ and  $\mu_{I|\tau_0}$:
\begin{equation}
  \begin{split}
	\mu_{R|\tau_0} =  E[R] = &\sum_{i=0}^{N-1}\bar{a}_iE[\bar{r}_i] - \sum_{i=0}^{N-1}\tilde{a}_iE[\tilde{r}_i]\\
	= &\sum_{i\in\Theta_{\tau_0}}\bar{a}_i2\sigma_s^2
  \end{split}
  \label{ER}
\end{equation}

\begin{equation}
  \begin{split}
	\mu_{I|\tau_0} =  E[I] = &\sum_{i=0}^{N-1}\tilde{a}_iE[\bar{r}_i] + \sum_{i=0}^{N-1}\bar{a}_iE[\tilde{r}_i]\\
	= &\sum_{i\in\Theta_0}\tilde{a}_i2\sigma_s^2\,.
  \end{split}
  \label{EI}
\end{equation}

The  second order statistic of $R$ and $I$ can be computed as
\begin{equation}
  \begin{split}
	E[R^2] = &E[(\sum_{i=0}^{N-1}\bar{a}_i\bar{r}_i - \sum_{i=0}^{N-1}\tilde{a}_i\tilde{r}_i)(\sum_{i=0}^{N-1}\bar{a}_i\bar{r}_i - \sum_{i=0}^{N-1}\tilde{a}_i\tilde{r}_i)]\\
	= &E[\sum_{i=0}^{N-1}\sum_{k=0}^{N-1}\bar{a}_i\bar{a}_k\bar{r}_i\bar{r}_k - \sum_{i=0}^{N-1}\sum_{k=0}^{N-1}\bar{a}_i\tilde{a}_k\bar{r}_i\tilde{r}_k - \sum_{i=0}^{N-1}\sum_{k=0}^{N-1}\bar{a}_i\tilde{a}_k\bar{r}_i\tilde{r}_k + \sum_{i=0}^{N-1}\sum_{k=0}^{N-1}\tilde{a}_i\tilde{a}_k\tilde{r}_i\tilde{r}_k]\\
	= &\sum_{i=0}^{N-1}\sum_{k=0}^{N-1}\bar{a}_i\bar{a}_kE[\bar{r}_i\bar{r}_k] - \sum_{i=0}^{N-1}\sum_{k=0}^{N-1}\bar{a}_i\tilde{a}_kE[\bar{r}_i\tilde{r}_k] - \sum_{i=0}^{N-1}\sum_{k=0}^{N-1}\bar{a}_i\tilde{a}_kE[\bar{r}_i\tilde{r}_k] + \sum_{i=0}^{N-1}\sum_{k=0}^{N-1}\tilde{a}_i\tilde{a}_kE[\tilde{r}_i\tilde{r}_k]\,.
  \end{split}
  \label{ER^2}
\end{equation}
Substitute the statistics of $r_i$ and eliminate the zero elements, we have
\begin{equation}
  \begin{split}
	E[R^2]  
	= &\sum_{i=0}^{N-1}(\bar{a}_i^2E[\bar{r}_i^2] + \tilde{a}_i^2E[\tilde{r}_i^2])\\
	= &\sum_{i\in\Theta_{\tau_0}}\bar{a}_i^2(8\sigma_{s_A}^4+4\sigma_{s_A}^2\sigma_n^2+2\sigma_n^4) + \sum_{i\in\Theta_{\tau_0}}\tilde{a}_i^2(4\sigma_{s_A}^2\sigma_n^2+2\sigma_n^4) + \sum_{i\notin\Theta_{\tau_0}}(\tilde{a}_i^2+\bar{a}_i^2)2(\sigma_n^2+\sigma_{s_A}^2)^2
  \end{split}<++>
  \label{<++>}
\end{equation}<++>
From the definition of $a_i$, we have
\begin{equation}
  \bar{a}_i^2 + \tilde{a}_i^2 =1
  \label{aisquare}
\end{equation}
hence \eqref{ER} can be written as
\begin{equation}
  \begin{split}
	E[R^2] = &\sum_{i\in\Theta_{\tau_0}}  (4\sigma_{s_A}^2\sigma_n^2+2\sigma_n^4)+ 8\sum_{i\in\Theta_{\tau_0}}\bar{a}_i^2\sigma_{s_A}^4 + 2\sum_{i\notin\Theta_{\tau_0}}(\sigma_n^2+\sigma_{s_A}^2)^2\\
	= &\#\Theta_{\tau_0}(4\sigma_{s_A}^2\sigma_n^2+2\sigma_n^4) +  8\sum_{i\in\Theta_{\tau_0}}\bar{a}_i^2\sigma_{s_A}^4+ 2(N - \#\Theta_{\tau_0})(\sigma_n^2+\sigma_{s_A}^2)^2 
  \end{split}
  \label{ER2}
\end{equation}
where $\#\Theta_{\tau_0}$ is the cardinality of set $\Theta_{\tau_0}$. Let $\sigma_{R|\tau_0}$ $\sigma_{I|\tau_0}$ denote the standard deviation of $R$ and $I$ when synchronization mismatch is $\tau_0$, 
\begin{equation}
  \begin{split}
	\sigma_{R|\tau_0} = &\sqrt{E[R^2] - E[R]^2}\\
	= &\left(\#\Theta_{\tau_0}(4\sigma_{s_A}^2\sigma_n^2+2\sigma_n^4) + 2(N - \#\Theta_{\tau_0})(\sigma_n^2+\sigma_{s_A}^2)^2 +  8\sum_{i\in\Theta_{\tau_0}}\bar{a}_i^2\sigma_{s_A}^4 - \mu_{R|\tau_0}^2\right)^\frac{1}{2}\,.
  \end{split}
  \label{deviationR}
\end{equation}
Similarly, we compute $\sigma_{I|\tau_0}$ and $E[RI]$ as
\begin{equation}
  \begin{split}
	E[I^2] = &E[(\sum_{i=0}^{N-1}\tilde{a}_i\bar{r}_i + \sum_{i=0}^{N-1}\bar{a}_i\tilde{r}_i)(\sum_{i=0}^{N-1}\tilde{a}_i\bar{r}_i + \sum_{i=0}^{N-1}\bar{a}_i\tilde{r}_i)]\\
	= &\sum_{i=0}^{N-1}\sum_{k=0}^{N-1}\tilde{a}_i\tilde{a}_kE[\bar{r}_i\bar{r}_j] + \sum_{i=0}^{N-1}\sum_{k=0}^{N-1}\tilde{a}_i\bar{a}_jE[\bar{r}_i\tilde{r}_j] +\sum_{i=0}^{N-1}\sum_{k=0}^{N-1}\tilde{a}_i\bar{a}_jE[\bar{r}_i\tilde{r}_j] + \sum_{i=0}^{N-1}\sum_{k=0}^{N-1}\bar{a}_i\bar{a}_jE[\tilde{r}_i\tilde{r}_j] \\
	= &\sum_{i=0}^{N-1}(\bar{a}_i^2E[\tilde{r}^2] + \tilde{a}_i^2E[\bar{r}_i^2])\\
	= &\#\Theta_{\tau_0}(4\sigma_{s_A}^2\sigma_n^2+2\sigma_n^4) + 2(N - \#\Theta_{\tau_0})(\sigma_n^2+\sigma_{s_A}^2)^2 +  8\sum_{i\in\Theta_{\tau_0}}\tilde{a}_i^2\sigma_{s_A}^4 
  \end{split}
  \label{EI^2}
\end{equation}
%\begin{equation}
%  E[I^2] = \sum_{i=0}^{N-1}(\bar{a}_i^2E[\tilde{r}^2] + \tilde{a}_i^2E[\bar{r}_i^2])
%  \label{EI2}
%\end{equation }
\begin{equation}
  \begin{split}
	\sigma_{I|\tau_0} = &\sqrt{E[I^2] - E[I]^2}\\
	= &\left(\#\Theta_{\tau_0}(4\sigma_{s_A}^2\sigma_n^2+2\sigma_n^4) + 2(N - \#\Theta_{\tau_0})(\sigma_n^2+\sigma_{s_A}^2)^2 +  8\sum_{i\in\Theta_{\tau_0}}\tilde{a}_i^2\sigma_{s_A}^4 - \mu_{I|\tau_0}^2	\right)^\frac{1}{2}\,.
  \end{split}
  \label{deviationI}
\end{equation}
\begin{equation}
  \begin{split}
	E[RI]= &E[\sum_{i=0}^{N-1} \sum_{k=0}^{N-1} \bar{a}_i\tilde{a}_k\bar{r}_i\bar{r}_k + 
	  \sum_{i=0}^{N-1} \sum_{k=0}^{N-1} \bar{a}_i\bar{a}_k\bar{r}_i\tilde{r}_k - 
	  \sum_{i=0}^{N-1} \sum_{k=0}^{N-1} \tilde{a}_i\tilde{a}_k\tilde{r}_i\bar{r}_k - 
	\sum_{i=0}^{N-1} \sum_{k=0}^{N-1} \tilde{a}_i\bar{a}_k\tilde{r}_i\tilde{r}_k]\\
	= &\sum_{i=0}^{N-1} \sum_{k=0}^{N-1} \bar{a}_i\tilde{a}_kE[\bar{r}_i\bar{r}_k] + 
	\sum_{i=0}^{N-1} \sum_{k=0}^{N-1} \bar{a}_i\bar{a}_kE[\bar{r}_i\tilde{r}_k] - 
	\sum_{i=0}^{N-1} \sum_{k=0}^{N-1} \tilde{a}_i\tilde{a}_kE[\tilde{r}_i\bar{r}_k] - 
	\sum_{i=0}^{N-1} \sum_{k=0}^{N-1} \tilde{a}_i\bar{a}_kE[\tilde{r}_i\tilde{r}_k]\\
	= &\sum_{i=0}^{N-1}\tilde{a}_i\bar{a}_iE[\bar{r}_i^2] - \sum_{i=0}^{N-1}\tilde{a}_i\bar{a}_iE[\tilde{r}_i^2]\\
	= &8\sigma_{s_A}^2\sum_{i\in \Theta_{\tau_0}}\tilde{a}_i\bar{a}_i
	\label{ERI}
  \end{split}
\end{equation}
The last step comes from the fact that when $i \notin \Theta_{\tau_0}$, $E[\bar{r}_i^2] = E[\tilde{r}_i^2] = 2(\sigma_{s_A}^2 + \sigma_{n}^2)^2$.
Let $\rho_{\tau_0}$ represents the correlation between $R$ and $I$ with synchronization mismatch $\tau_0$, we have
\begin{equation}
  \begin{split}
	\rho_{\tau_0} = &\frac{E[RI]-\mu_{R|\tau_0}\mu_{I|\tau_0}}{\sigma_{R|\tau_0}\sigma_{I|\tau_0}}\\
	= &\frac{8\sigma_{s_A}^2\sum_{i\in \Theta_{\tau_0}}\tilde{a}_i\bar{a}_i - \mu_{R|\tau_0}\mu_{I|\tau_0}}{\sigma_{R|\tau_0}\sigma_{I|\tau_0}}
  \end{split}
  \label{RIcorrelation}
\end{equation}
By substituting \eqref{ER} \eqref{EI}, \eqref{deviationR} and \eqref{deviationI} into \eqref{RIcorrelation}, we can get the value of $\rho_{\tau_0}$. 
The distribution of 
$\begin{bmatrix}
  R \\
  I
\end{bmatrix}$ with synchronization mismatch $\tau = \tau_0$ can be written as
\begin{multline}
  f(R,I | \tau=\tau_0) = \frac{1}{2\pi\sigma_{I|\tau_0}\sigma_{R|\tau_0}\sqrt{1 - \rho_{\tau_0}^2}}\exp\left( -\frac{1}{2(1-\rho_{\tau_0}^2)}\left[ \frac{(R-\mu_{R|\tau_0})^2}{\sigma_{R|\tau_0}^2} + \right.\right.\\
  \left. \left.  \frac{(I-\mu_{I|\tau_0})^2}{\sigma_{I|\tau_0}^2} - \frac{2\rho(R-\mu_{R|\tau_0})(I-\mu_{I|\tau_0})}{\sigma_{R|\tau_0}\sigma_{I|\tau_0}}\right] \right)
  \label{disRI}
\end{multline}

In practise, the synchronization mismatch is governed by discrete niform distribution between $0$ and $l_0-1$. Hence in the situation of unknown $\tau$, the distribution of  
$\begin{bmatrix}
  R \\
  I
\end{bmatrix}$
under hypothesis $H_1$ can be written in form of 
\begin{equation}
  \begin{split}
	f_1(R, I) &= \sum_{\tau_0 = 0}^{l_0-1} \frac{1}{l_0}f(R, I|\tau=\tau_0)\\
	&= \sum_{\tau_0 = 0}^{l_0-1}\frac{1}{2l_0\pi\sigma_{I|\tau_0}\sigma_{R|\tau_0}\sqrt{1 - \rho_{\tau_0}^2}}\exp\left( -\frac{1}{2(1-\rho_{\tau_0}^2)}\left[ \frac{(R-\mu_{R|\tau_0})^2}{\sigma_{R|\tau_0}^2} + \right.\right.\\
	& \left. \left. \;\;\;\;\;\;\;\;\;\;\;\;\;\;\;\; \;\;\;\;\;\;\;\;\;\;\;\;\;\;\;\; \;\;\;\;\;\;\;\;\;\;\;\;\;\;\;\; \frac{(I-\mu_{I|\tau_0})^2}{\sigma_{I|\tau_0}^2} - \frac{2\rho(R-\mu_{R|\tau_0})(I-\mu_{I|\tau_0})}{\sigma_{R|\tau_0}\sigma_{I|\tau_0}}\right] \right)
  \end{split}
  \label{f_1underH1}
\end{equation}
and the statistic of $R$ and $I$  under hypothesis $H_1$ are summarized in Table \ref{Table2} on page \pageref{Table2}.
\begin{table}[h]
  \begin{tabular}{|c|c|}
	\hline
	Statistics          & Value                                                                                                                                                                                                                 \\ \hline
	$\mu_{R|\tau_0}$    & $\displaystyle{\sum_{i\in\Theta_{\tau_0}}\bar{a}_i2\sigma_{s_A}^2}$                                                                                                                                                                      \\ \hline
	$\mu_{I|\tau_0}$    & $\displaystyle{\sum_{i\in\Theta_0}\tilde{a}_i2\sigma_{s_A}^2}$                                                                                                                                                                           \\ \hline
	$\sigma_{R|\tau_0}$ & $\displaystyle{\left(\#\Theta_{\tau_0}(4\sigma_{s_A}^2\sigma_n^2+2\sigma_n^4) + 2(N - \#\Theta_{\tau_0})(\sigma_n^2+\sigma_{s_A}^2)^2+8\sum_{i\in\Theta_{\tau_0}}\bar{a}_i^2\sigma_{s_A}^4 - \mu_{R|\tau_0}^2 \right)^\frac{1}{2}}$  \\ \hline
	$\sigma_{I|\tau_0}$ & $\displaystyle{\left(\#\Theta_{\tau_0}(4\sigma_{s_A}^2\sigma_n^2+2\sigma_n^4) + 2(N - \#\Theta_{\tau_0})(\sigma_n^2+\sigma_{s_A}^2)^2+8\sum_{i\in\Theta_{\tau_0}}\tilde{a}_i^2\sigma_{s_A}^4 - \mu_{R|\tau_0}^2\right)^\frac{1}{2}}$ \\ \hline
	$\rho_{\tau_0}$    & $ \displaystyle{\frac{8\sigma_{s_A}^2\sum_{i\in \Theta_{\tau_0}}\tilde{a}_i\bar{a}_i - \mu_{R|\tau_0}\mu_{I|\tau_0}}{\sigma_{R|\tau_0}\sigma_{I|\tau_0}}}$                                                                            \\ \hline
  \end{tabular}
  \caption{Statistic for $R$ and $I$}
  \label{Table2}
\end{table}

Similarly the distribution of  
$\begin{bmatrix}
  R \\
  I
\end{bmatrix}$
under hypothesis $H_1$ can be written in form of 
\begin{equation}
  \begin{split}
	f_2(R, I) &= \sum_{\tau_0 = 0}^{l_0-1} \frac{1}{l_0}f(R, I|\tau=\tau_0)\\
	&= \sum_{\tau_0 = 0}^{l_0-1}\frac{1}{2l_0\pi\sigma_{I|\tau_0}\sigma_{R|\tau_0}\sqrt{1 - \rho_{\tau_0}^2}}\exp\left( -\frac{1}{2(1-\rho_{\tau_0}^2)}\left[ \frac{(R-\mu_{R|\tau_0})^2}{\sigma_{R|\tau_0}^2} + \right.\right.\\
	& \left. \left. \;\;\;\;\;\;\;\;\;\;\;\;\;\;\;\; \;\;\;\;\;\;\;\;\;\;\;\;\;\;\;\; \;\;\;\;\;\;\;\;\;\;\;\;\;\;\;\; \frac{(I-\mu_{I|\tau_0})^2}{\sigma_{I|\tau_0}^2} - \frac{2\rho(R-\mu_{R|\tau_0})(I-\mu_{I|\tau_0})}{\sigma_{R|\tau_0}\sigma_{I|\tau_0}}\right] \right)\,,
  \end{split}
  \label{f_1underH1}
\end{equation}
wherestatistics of $R$ and $I$ under hypothesis $H_2$ is given in Table \ref{Table3} on page \pageref{Table3}. 

\begin{table}[h] 
  \begin{tabular}{|c|c|}
	\hline
	Statistics          & Value                                                                                                                                                                                                                 \\ \hline
	$\mu_{R|\tau_0}$    & $\displaystyle{\sum_{i\in\Theta_{\tau_0}}\bar{a}_i2\sigma_{s_B}^2}$                                                                                                                                                                      \\ \hline
	$\mu_{I|\tau_0}$    & $\displaystyle{\sum_{i\in\Theta_0}\tilde{a}_i2\sigma_{s_B}^2}$                                                                                                                                                                           \\ \hline
	$\sigma_{R|\tau_0}$ & $\displaystyle{\left(\#\Theta_{\tau_0}(4\sigma_{s_B}^2\sigma_n^2+2\sigma_n^4) + 2(N - \#\Theta_{\tau_0})(\sigma_n^2+\sigma_{s_B}^2)^2+8\sum_{i\in\Theta_{\tau_0}}\bar{a}_i^2\sigma_{s_B}^4 - \mu_{R|\tau_0}^2 \right)^\frac{1}{2}}$  \\ \hline
	$\sigma_{I|\tau_0}$ & $\displaystyle{\left(\#\Theta_{\tau_0}(4\sigma_{s_B}^2\sigma_n^2+2\sigma_n^4) + 2(N - \#\Theta_{\tau_0})(\sigma_n^2+\sigma_{s_B}^2)^2+8\sum_{i\in\Theta_{\tau_0}}\tilde{a}_i^2\sigma_{s_B}^4 - \mu_{R|\tau_0}^2\right)^\frac{1}{2}}$ \\ \hline
	$\rho_{\tau_0}$    & $ \displaystyle{\frac{8\sigma_{s_B}^2\sum_{i\in \Theta_{\tau_0}}\tilde{a}_i\bar{a}_i - \mu_{R|\tau_0}\mu_{I|\tau_0}}{\sigma_{R|\tau_0}\sigma_{I|\tau_0}}}$                                                                            \\ \hline
  \end{tabular}
  \caption{Statistic for $R$ and $I$ under hypothesis $H_2$}
  \label{Table3}
\end{table}

Next we consider the distribution of 
$\begin{bmatrix}
  R \\
  I
\end{bmatrix}$
under hypothesis $H_0$. 
According to the expression of $R$ and $I$ given in \eqref{R} and \eqref{I}, $\mu_R$ and $\mu_I$ can be computed as
\begin{equation}
  \begin{split}
	\mu_R =E[R] = &\sum_{i=0}^{N-1}\bar{a}_iE[\bar{r}_i] - \sum_{i=0}^{N-1}\tilde{a}_iE[\tilde{r}_i] \\
	= &0
  \end{split}
  \label{ERnoise}
\end{equation}
and 
\begin{equation}
  \begin{split}
	\mu_{I|\tau_0} =  E[I] = &\sum_{i=0}^{N-1}\tilde{a}_iE[\bar{r}_i] + \sum_{i=0}^{N-1}\bar{a}_iE[\tilde{r}_i]\\
	= &0\,.
  \end{split}
  \label{EInoise}
\end{equation}
The second order statistics of $R$ and $I$ are
\begin{equation}
  \begin{split}
	E[R^2] = &E[(\sum_{i=0}^{N-1}\bar{a}_i\bar{r}_i - \sum_{i=0}^{N-1}\tilde{a}_i\tilde{r}_i)(\sum_{i=0}^{N-1}\bar{a}_i\bar{r}_i - \sum_{i=0}^{N-1}\tilde{a}_i\tilde{r}_i)]\\
	= &E[\sum_{i=0}^{N-1}\sum_{k=0}^{N-1}\bar{a}_i\bar{a}_k\bar{r}_i\bar{r}_k - \sum_{i=0}^{N-1}\sum_{k=0}^{N-1}\bar{a}_i\tilde{a}_k\bar{r}_i\tilde{r}_k - \sum_{i=0}^{N-1}\sum_{k=0}^{N-1}\bar{a}_i\tilde{a}_k\bar{r}_i\tilde{r}_k + \sum_{i=0}^{N-1}\sum_{k=0}^{N-1}\tilde{a}_i\tilde{a}_k\tilde{r}_i\tilde{r}_k]\\
	= &\sum_{i=0}^{N-1}\sum_{k=0}^{N-1}\bar{a}_i\bar{a}_kE[\bar{r}_i\bar{r}_k] - \sum_{i=0}^{N-1}\sum_{k=0}^{N-1}\bar{a}_i\tilde{a}_kE[\bar{r}_i\tilde{r}_k] - \sum_{i=0}^{N-1}\sum_{k=0}^{N-1}\bar{a}_i\tilde{a}_kE[\bar{r}_i\tilde{r}_k] + \sum_{i=0}^{N-1}\sum_{k=0}^{N-1}\tilde{a}_i\tilde{a}_kE[\tilde{r}_i\tilde{r}_k]\\
	= &\sum_{i=0}^{N-1}(\bar{a}_i^2E[\bar{r}_i^2] + \tilde{a}_i^2E[\tilde{r}_i^2])\\
	= &2N\sigma_n^4\,,
  \end{split}
  \label{ER^2noise}
\end{equation}

\begin{equation}
  \begin{split}
	E[I^2] = &E[(\sum_{i=0}^{N-1}\tilde{a}_i\bar{r}_i + \sum_{i=0}^{N-1}\bar{a}_i\tilde{r}_i)(\sum_{i=0}^{N-1}\tilde{a}_i\bar{r}_i + \sum_{i=0}^{N-1}\bar{a}_i\tilde{r}_i)]\\
	= &\sum_{i=0}^{N-1}\sum_{k=0}^{N-1}\tilde{a}_i\tilde{a}_kE[\bar{r}_i\bar{r}_j] + \sum_{i=0}^{N-1}\sum_{k=0}^{N-1}\tilde{a}_i\bar{a}_jE[\bar{r}_i\tilde{r}_j] +\sum_{i=0}^{N-1}\sum_{k=0}^{N-1}\tilde{a}_i\bar{a}_jE[\bar{r}_i\tilde{r}_j] + \sum_{i=0}^{N-1}\sum_{k=0}^{N-1}\bar{a}_i\bar{a}_jE[\tilde{r}_i\tilde{r}_j] \\
	= &\sum_{i=0}^{N-1}(\bar{a}_i^2E[\tilde{r}^2] + \tilde{a}_i^2E[\bar{r}_i^2])\\
	= &2N\sigma_n^4
  \end{split}
  \label{EI^2noise}
\end{equation}
and

\begin{equation}
  \begin{split}
	E[RI]= &E[\sum_{i=0}^{N-1} \sum_{k=0}^{N-1} \bar{a}_i\tilde{a}_k\bar{r}_i\bar{r}_k + 
	  \sum_{i=0}^{N-1} \sum_{k=0}^{N-1} \bar{a}_i\bar{a}_k\bar{r}_i\tilde{r}_k - 
	  \sum_{i=0}^{N-1} \sum_{k=0}^{N-1} \tilde{a}_i\tilde{a}_k\tilde{r}_i\bar{r}_k - 
	\sum_{i=0}^{N-1} \sum_{k=0}^{N-1} \tilde{a}_i\bar{a}_k\tilde{r}_i\tilde{r}_k]\\
	= &\sum_{i=0}^{N-1} \sum_{k=0}^{N-1} \bar{a}_i\tilde{a}_kE[\bar{r}_i\bar{r}_k] + 
	\sum_{i=0}^{N-1} \sum_{k=0}^{N-1} \bar{a}_i\bar{a}_kE[\bar{r}_i\tilde{r}_k] - 
	\sum_{i=0}^{N-1} \sum_{k=0}^{N-1} \tilde{a}_i\tilde{a}_kE[\tilde{r}_i\bar{r}_k] - 
	\sum_{i=0}^{N-1} \sum_{k=0}^{N-1} \tilde{a}_i\bar{a}_kE[\tilde{r}_i\tilde{r}_k]\\
	= 0\,.
	\label{ERInoise}
  \end{split}
\end{equation}
The standard deviation of $R$ and $I$ under hypothesis $H_0$ is
\begin{equation}
  \sigma_R = \sqrt{E[R^2] - E[R]^2} = \sqrt{2N}\sigma_n^2
  \label{deviationR}
\end{equation}
and
\begin{equation}
  \sigma_I = \sqrt{E[I^2] - E[I]^2} = \sqrt{2N}\sigma_n^2
  \label{deviationR}
\end{equation}

Further more, since $E[RI] = E[R]E[I]$ = 0 and $R, I$ are combined gaussian distributed,  we know under hypothesis $H_0$ they are independent. Hence the distribution of $\begin{bmatrix}
  R \\
  I
\end{bmatrix}$ under hypothesis $H_0$ can be written as
\begin{equation}
  \begin{split}
	f_0(R, I)= &f_0(R)f_0(I)\\
	= &\frac{1}{\sqrt{2N}\sigma_n^2\sqrt{2\pi}}\exp\left( -\frac{R^2}{4N\sigma_n^4} \right)\frac{1}{\sqrt{2N}\sigma_n^2\sqrt{2\pi}}\exp\left( -\frac{R^2}{4N\sigma_n^4} \right)\\
	=&\frac{1}{4\pi N\sigma_n^4}\exp\left( -\frac{I^2+R^2}{4N\sigma_n^4} \right) 
  \end{split}
  \label{f0RO}
\end{equation}
\subsection{Numerical Results}

