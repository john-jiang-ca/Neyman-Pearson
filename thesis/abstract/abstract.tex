\documentclass{report}
\usepackage{amssymb}
\usepackage{mathtools}
\usepackage{ amsmath, bm }
\usepackage{graphicx}
\usepackage{epstopdf}
\usepackage{yhmath}
\DeclareGraphicsExtensions{.pdf,.eps,.png,.jpg,.mps}
\usepackage{dsfont}
\usepackage{bbm}
\usepackage{setspace}
\usepackage[top=1.5in, bottom=1.5in, left=1in, right=1in]{geometry}

%For lemma, theorem, proposition, corllary, proof, definition,example and remark.
\newcommand{\rmnum}[1]{\romannumeral #1}
\newcommand{\Rmnum}[1]{\expandafter\@slowromancap\romannumeral #1@}
\makeatother

\newcommand\bigfrown[2][\textstyle]{\ensuremath{%
  \array[b]{c}\text{\scalebox{2}{$#1\frown$}}\\[-1.3ex]#1#2\endarray}}
% End
\author{An~Jiang,~
        Harry~Leib,~\\
          Department of Electrical Engineering\\
          McGill University\\
          Montreal,~Quebec,~Canada,~\\
          Email: an.jiang@mail.com}
\title{The Extended Neyman-Pearson Hypotheses Testing Framework and its Application to Spectrum Sensing in Cognitive Wireless Communication}
\date{\today}
\begin{document}
\begin{spacing}{2.0}
\maketitle
\begin{abstract}
We proposed a Modified Extended Neyman Pearson test that is suitable for spectrum sensing when there might be different type of primary signals. Our work generalizes existing Neyman Pearson (NP) test to multiple probability of false alarm constraints. This allows the Cognitive Radio wireless communication system to provide different levels of protection to various of primary users.  
M-ROC surface, figure concerning the relationship between the probability of detection and the constraint conditions, is developed to depict the performance of MENP testing. 

Using MENP framework, the optimal decision rule of energy based spectrum sensing detector for OFDM signals under AWGN channel was derived in term of inverse CDF of Chi-Square distribution.   

Computer simulations were performed to study the M-ROC surface of energy based detector. [TO BE ADDED] 
\end{abstract}
\section{Introduction}
[CHOOSE ONE AS THE FIRST PARAGRAPH]
Future wireless communication systems will require higher transmission speed in order to deliver large amount of data \cite{pelcat20133gpp}. The main obstacle for providing high speed wireless communication services is the limitation of available frequency bandwidth. In order to make potential radio system use spectrum more efficiently than in the past, advanced technologies are creating, such as Cognitive Radio \cite{nonotice}. 

Wireless communication industry has been growing fast in the past decade. As a result, the data transmission rate of traditional wireless communication services have been largely increased (e.g., more and more cellphone companies adopted LET system \cite{cox2014introduction}). In addition, new wireless services like wireless sensor network are emerging from research to practice. With the growing number of wireless communication systems and the increasing throughputs of each wireless communication system, the available radio spectrum has become limited \cite{mitola1999cognitive} \cite{federal2005notice}. However, the actual measurements carried out by Federal Communications Commission (FCC) shows that frequency spectrum is inefficiently utilized with the range of 15\% ~ 85\% with wide variance in time and space \cite{nonotice}. Hence the root cause of spectrum scarcity is not the physical shortage of spectrum rather the inefficient spectrum usage \cite{umar2013comparative}. 
In order to make potential radio system use spectrum more efficiently than in the past, advanced technologies are creating, such as Cognitive Radio \cite{nonotice}.

Cognitive radio is an intelligent wireless communication system that is aware of its surrounding environment and uses of the methodology of understanding -by-building to learn from the environment and adapt its internal states to statistical variations in the incoming RF stimuli by making corresponding changes in certain operating parameters in real-time, with two primary objectives in mind: 1. highly reliable communications whenever and wherever needed; 2. efficient utilization of the radio spectrum \cite{a001}. In cognitive radio system, users are divided into two types: primary user and secondary user. A primary user is legacy user or a licensed user who has priority on the particular part of frequency band.  Potential primary user may include cellphone communication system (e.g., GSM 3G or LTE), Digital TV broadcast system, microphone communication system and so on. A secondary user is defined as a cognitive user who has lower rights to access the particular spectrum. A secondary user can access the spectrum when the particular frequency is idle but has to vacate the frequency band as soon as the primary become active in order not to cause interference to primary user. In order to achieve this goal, spectrum sensing component is introduced into cognitive radio communication system \cite{buddhikot2007understanding} \cite{tandra2009spectrum}.   

Spectrum sensing is the task of identifying vacant band under certain radio frequency environment and detecting the primary users with high probability of detection as soon as they become active \cite{umar2012spectrum}. According to the number of detector, spectrum sensing can be divided into single-user sensing (local detection) \cite{axell2010overview} \cite{wang2011advances} and cooperative sensing \cite{akyildiz2011cooperative}. Even though single user sensing needs only one detector, its performance decreases heavily in challenging propagation environment (e.g. multipath fading channel, shadowing). In such scenario, cooperative sensing is preferred.  
According to a priori information required by spectrum sensing detector, spectrum sensing techniques can be categorized as non-blind, semi-blind and blind. Non-blind sensing require the primary user's signal as well as the noise power to primary user's activity. Semi-blind scheme need a priori information about the variance of the noise to determine the status of the frequency band interested. Blind sensing can detect the primary user's activities in particular frequency band without any priori information. 
Common techniques for spectrum sensing detector are based on energy detection, exploitation of cyclostationarity properties of the signal being sensed, and also could user preamble sequences that are embedded in the signals \cite{cabric2004implementation}. In \cite{axell2011optimal}, the authors point out that for OFDM signals when the variances of the noise and signal are known, the performance of energy based spectrum sensing is very close to optimal. Cooperative spectrum sensing \cite{ganesan2005cooperative}, employing multiple sensors that are geographically scattered, are effective against shadowing.
Different sensing algorithms and DSP techniques have been introduced for specific situations, such as in \cite{tian2007compressed} where the authors study algorithms for wide band spectrum sensing focusing on sub-Neyquist sampling techniques. In  \cite{sun2013wideband} and  \cite{sun2013wideband2} the authors propose compressed sensing and sub-Nyquist sampling techniques for wide band spectrum sensing.

In order to detect a free frequency channel, a spectrum sensing scheme solves a binary hypotheses testing problem, where the null hypothesis refers to the event that a user is not using the channel and the alternative hypothesis refers to the event that a user is occupying the channel. If the a-prior probabilities of null and alternative hypotheses are known, then such a problem can be solved using a Bayesian framework \cite{poor1994introduction}. The situation when Baysian framework could be used is considered in \cite{zeng2010review} and the performance of such scheme is analyzed.
Even though the Baysian framework can be used in some situations, in more often cases, the a-prior probabilities are not available. In such situations Neyman Pearson testing is employed \cite{poor1994introduction}.When there is a single type of primary user (two hypotheses), NP testing will achieve the largest probability of detecting a vacant channel when no primary user is present under a constraint on the probability that the channel will be declared vacant when in fact a primary user is present.
Traditional NP testing has been extensively used for two hypotheses problem. In such cases,  the performance is characterized by the Receiver Operating Characteristic (ROC) curve, which represents the relationship between probability of detection and probability of false alarm \cite{poor1994introduction}.

In spectrum sensing application, there may be more than one type of primary users. For example in IEEE 802.22 \cite{shellhammer2008spectrum} the channel can be occupied by Analog-TV Digital-TV and wireless microphone. In such cases, an extended NP (ENP) test can provide important advantages \cite{zhang1999design}. A detector based on ENP testing can ensure the largest probability of detection under a separate constraint of false alarm for each primary user type \cite{LehmannTest}.

With an ENP detector, the achievable false alarm probabilities are limited. In this work we analyze the properties of the ENP test and propose the Modified Extended Neyman Pearson (MENP) test, that circumvents the limitations of the false alarm probabilities. The structure of the remaining of this paper is [TO BE ADDED].

\newpage
\bibliographystyle{ieeetr}	% (uses file "plain.bst")
\bibliography{mybib}		% expects file "myrefs.bib"
\end{spacing}
\end{document}
