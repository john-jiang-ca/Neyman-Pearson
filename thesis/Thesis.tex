\documentclass [12pt,letterpaper]{report}

% Standard packages
\usepackage{amsmath}		% Extra math definitions
\usepackage{graphics}		% PostScript figures
\usepackage{setspace}		% 1.5 spacing
\usepackage{longtable}          % Tables spanning pages

% Custom packages
\usepackage[first]{datestamp}	% Datestamp on first page of each chapter
\usepackage[fancyhdr]{McECEThesis}	% Thesis style
\usepackage{McGillLogo}		% McGill University crest

% $Id: ThesisEx.tex,v 1.1 2005/06/09 12:48:46 kabal Exp $

\usepackage{color}
\def\headrulehook{\color{red}}		% Color the header rule

%===== page layout
% Define the side margins for a right-side page
\insidemargin = 1.3in
\outsidemargin = 0.8in

% Above margin is space above the header
% Below margin is space below footer
\abovemargin = 1.1in
\belowmargin = 0.75in

%========= Document start

\begin {document}

%===== Title page

\title{The Extended Neyman-Pearson Hypotheses Testing Framework and its Application to Spectrum Sensing in Cognitive Wireless Communication}
\author{An Jiang}
\date{\Month\ \number\year}
\organization{%
  \\[0.2in]
  \McGillCrest {!}{1in}\\	% McGill University crest
  \\[0.1in]
  Department of Electrical \& Computer Engineering\\
  McGill University\\
  Montreal, Canada}
\note{%
  {\color{red} \hrule height 0.4ex}
  \vskip 3ex
  A thesis submitted to McGill University in partial fulfillment of the
  requirements for the degree of Whatever.
  \vskip 3ex
  \copyright\ \the\year\ An Jiang
}

\maketitle

%===== Justification, spacing for the main text
\raggedbottom
\onehalfspacing
\pagenumbering{roman}

%===== Abstract, Sommaire & Acknowledgments
\section*{\centering Abstract}
We propose a Modified Extended Neyman Pearson (MENP) framework which is suitable for spectrum sensing when there might be different types of primary signals. Our work generalizes existing Neyman Pearson (NP) test to multiple probability of false alarm constraints. This allows the cognitive radio wireless communication system to provide different levels of protection for various primary users.  
M-ROC surface, representing the relationship between the probability of detection and the constraints, is used to depict the performance of MENP testing.   

Using MENP framework, an energy based spectrum detector and a cyclostationary based spectrum detector, are developed to detect OFDM signals. These detectors are designed to have the largest probability of detection under any probability of false alarm constraints.  

Performance analyses were performed to study the M-ROC surface behavior for both detectors. We then evaluate the improvement of the probability of detection with respect to the increase of probability of false alarms.  
The performance analysis also confirms that the optimal decision rule of energy based spectrum detector can be derived in terms of the inverse CDF of Chi-Square distribution.  

\newpage

\section*{\centering Acknowledgments}

Thesis regulations require that contributions by others in the collection of
 materials and data, the design and construction of apparatus, the performance
 of experiments, the analysis of data, and the preparation of the thesis be
 acknowledged.

%========== Tables of contents, figures, tables
\tableofcontents
\listoffigures
\listoftables

\newpage
\chapter*{List of Acronyms}\markright{List of Terms}

\begin{longtable}{ll}
  16-QAM   &  16-point Quadrature Amplitude Modulation\\
  3GPP     &  Third Generation Partnership Project\\
  3GPP2    &  Third Generation Partnership Project 2\\
  64-QAM   &  64-point Quadrature Amplitude Modulation\\
  ADSL     &  Asymmetric Digital Subscriber Line\\
  ARQ      &  Automatic Repeat Request\\
  WPAN     &  Wireless Personal Area Network
\end{longtable}

\cleardoublepage
\pagenumbering{arabic}

%========== Chapters
\typeout{}
\chapter{Introduction}
Future wireless communication systems will require higher transmission speed in order to deliver large amount of data \cite{pelcat20133gpp}. The main obstacle for providing high speed wireless communication services is the limitation of available radio frequency resources. In order to make radio systems use the radio spectrum more efficiently than in the past, advanced technologies are created, such as Cognitive Radio (CR) \cite{federal2005notice}. 

A CR is an intelligent wireless communication system which is able to monitor its operating spectrum environments and change its transmitter parameters to best match these conditions \cite{wang2011advances, a001}. It provides a novel approach for the coexistence of different wireless communication systems in the same frequency band. In a CR communication system, users are divided into two types: primary users and secondary users. A primary user is a legacy user, or a licensed user, who has priority on the particular frequency band. The CR system must ensure highly reliable communications whenever and wherever needed by primary users \cite{a001}. Potential primary users may include cellphone communication systems (e.g., GSM, CDMA or LTE), Satellite TV Broadcast systems, microphone communication systems etc.. A secondary user, or a cognitive user, is referred to someone who has lower priority to access the particular frequency band. A secondary user can utilize the spectrum only when the particular frequency is idle, and has to vacate the frequency band as soon as any primary user becomes active. 
In order to monitor the utilization of the spectrum in real-time, a spectrum sensing functionality must be introduced into a CR communication system \cite{buddhikot2007understanding, tandra2009spectrum}.   

Spectrum sensing refers to the task of identifying vacant frequency channel in a certain radio frequency band and detecting the primary users with high probability \cite{umar2012spectrum}. 
%A spectrum sensing system includes at least one measuring device and at least one testing device. The role of measuring device is transferring the RF stimuli into a suitable test statistics. The role of testing device is decide the spectrum state according to the output(s) of measuring device(s).
Spectrum sensing system can be divided into single-user sensing (local detection) and cooperative sensing \cite{wang2011advances, akyildiz2011cooperative, ma2008soft, axell2010overview}. In single-user sensing, the system makes a decision about the status of a frequency channel based on the observation from one cognitive device. The main advantage of single user sensing is its low hardware complexity. However, it has been shown that its performance deteriorates significantly in challenging propagation environments (e.g. multipath fading channel, shadowing) \cite{akyildiz2011cooperative}. In such scenarios, cooperative sensing is preferred. Cooperative sensing is a detection methods where multiple sensing device cooperate to detect the status of a radio frequency channel \cite{ganesan2005cooperative, arslan2007cognitive}. While cooperative sensing improves detection performance in challenging environments, it requires a more complicated system design,  longer sensing time,  and increased energy consumption \cite{akyildiz2011cooperative}.  
Moreover, the communication between sensing devices can lead to extra spectrum usage. 
Various sensing methods have been considered for identifying spectrum usage opportunities. Main sensing techniques include  energy detection, cyclostationary detection and waveform detection. 

Energy based detection is the simplest spectrum sensing techniques, that has been extensively used in radiometers \cite{cabric2004implementation, poor1994introduction, urkowitz1967energy}. It employs the energy of received signal to identify the channel status. Since it does not require complicated digital signal processing, it has low energy consumption and low hardware complexity.  Another favorable aspect of an energy detector is that it does not require  any information about the primary user's signal.  
The shortcomings of an energy detector include: (1) the long time to detect a  primary user if the SNR is low; (2) requires information about the variance of noise, which in some case is hard to estimate; (3) performance is not satisfying in multipath channels \cite{akyildiz2011cooperative}; (4) unable to detect spread spectrum signals like CDMA \cite{urkowitz1967energy, akyildiz2011cooperative}. 
In \cite{akyildiz2011cooperative} the authors show that the first three problems can be mitigated by diversity gain from cooperation,  making it a suitable techniques in cooperative sensing for non-CDMA signals.  

Cyclostationarity feature based detector identifies the spectrum status by exploiting autocorrelation properties of the received signals, induced by its periodicity features \cite{goldsmith2009breaking}. Such periodicity features are induced by  sinusoidal carriers,  regularly transmitted pulse trains, pilot sequences and cyclic prefixes (CP) in OFDM signals \cite{akyildiz2011cooperative, umar2013comparative}.  
Comparison with an energy detection, a cyclostationary based detector is more robust to noise uncertainty and has better performance in the low SNR regime \cite{umar2013comparative}. However since a cyclostationarity detector requires high timing accuracy when sampling, the hardware is more complex than that of an energy detector \cite{yucek2009survey}. Another disadvantage of cyclostationarity based detector is that its performance over fading channels is not satisfying \cite{tandra2007snr}.   

Waveform based detection identifies the channel status by exploiting known patterns in the primary user's signal, such as pilot sequences. The favourable aspect of such a detector is that it can achieve a high probability of detection in a comparatively short time and is able to provide  good performance in a shadowing environment \cite{tang2005some}. However, since the detector needs to extract the pilot sequence, it relies on dedicated synchronization and accurate information about the primary user's signal structure.  

%Matched filter based detector has the similar structure as a primary user receiver. Even in low SNR environment, it can still achieve a relatively high probability of detection in a short detection time \cite{tandra2005fundamental}. However since it needs to demodulate the received signal, it requires perfect information of primary signal \cite{yucek2009survey}. Another unfavorable aspect of this detector is its high power consumption, which results from its complicated hardware structure.  

%Recent years, new sensing algorithms and digital signal processing techniques have been introduced for specific spectrum sensing situations. One of them is wavelets based detection \cite{tian2007compressed, sun2013wideband, sun2013wideband2}. In wavelets based detection, sub-Neyquist sampling and compressed sensing are used to estimate the power spectrum density in a wide band channel. After that it exploits wavelets to detect the power spectrum density (PSD) edges, which correspond to transition from a occupied status to a idle status, and to estimate the power between each edges. In this way, the wideband spectrum utilization is known and the secondary users can exploit one or several idle sub-bands to communicate. 

In order to detect a free frequency channel, a spectrum sensing scheme solves a binary hypotheses testing problem, where one hypothesis refers to the event that a user is not using the channel and the other hypothesis refers to the event that a user is occupying the channel. If the a-prior probabilities of null and alternative hypotheses are known, then such a problem can be solved by using a Bayesian framework \cite{poor1994introduction}. The situation when Baysian framework could be used is considered in \cite{zeng2010review}, where  the performance of such scheme is analyzed.
Even though the Baysian framework can be used in some situations, in more often cases, the a-prior probabilities are not available. In such situations Neyman Pearson testing is employed \cite{poor1994introduction}.When there is a single type of primary user (two hypotheses), NP testing will achieve the largest probability of detecting a vacant channel when no primary user is present under a constraint on the probability that the channel will be declared vacant when in fact a primary user is present.
Traditional NP testing has been extensively used for two hypotheses problems. In such cases,  the performance is characterized by the Receiver Operating Characteristic (ROC) curve, which represents the relationship between probability of detection and probability of false alarm \cite{poor1994introduction}. 

\section{Thesis Objectives}
In spectrum sensing application, there may be more than one type of primary users. For example in IEEE 802.22  the channel can be occupied by Analog-TV Digital-TV and wireless microphone\cite{shellhammer2008spectrum}. 
In such cases, an extended NP (ENP) test  \cite{LehmannTest} can provide important advantages. A detector based on ENP testing can ensure the largest probability of detection under a separate constraint of false alarm for each primary user type \cite{LehmannTest}.
With an ENP detector, the achievable false alarm probabilities are limited. This motivates us to analyze the properties of the ENP test and propose the Modified Extended Neyman Pearson (MENP) test, that circumvents the limitations of the false alarm probabilities. Furthermore, we also consider some theoretical  findings for spectrum sensing applications.  

\section{Thesis Organization and Contribution}
The Thesis is organized as follows. We review the ENP framework and propose the MENP framework in Chapter 2. In particular,  we consider  a necessary condition proof for an  ENP test followed by  two lemmas concerning the ROC surface of ENP tests. 
We then presents Modified Extended Neyman Pearson framework aiming to  achieve the largest probability of detection  under any probability of false alarm constraints.
Then an algorithm which aims at achieving the MENP parameters is reviewed. 

In Chapter 3, M-ROC, the figure depicting the relationship between probability of detection and probability of false alarm constraints, is considered.  We then present two examples of M-ROC surfaceone associated with a Gaussian problem and the other with a Chi-Square problem. 

Once the MENP framework is presented, we apply MENP framework to energy based detection and cyclostationary based detection in Chapter 4. We consider  performances of both detectors for the situation when there are two primary users.  

Chapter 5 provides a summary of this thesis, and recommends some directions for future work.  

The contributions of this thesis are:

$\bullet$ Proved a necessary condition for the Extended Neyman Pearson Lemma (\rmnum{3}) when the probability density functions (PDFs) of given hypotheses satisfy certain constraints. 

$\bullet$ Derived the normal direction vector for a point on the ROC surface. 

$\bullet$ Proposed a novel detection scheme MENP test, which generalizes the Extended Neyman Pearson test to  achieve the largest probability of detection under any probability of false alarm constraints. 

$\bullet$ Developed numerical results to show the performance of MENP test under Gaussian situations and Chi-Square situations. 

$\bullet$ Using MENP test, energy based detector and cyclostationary based detector are derived to detect spectrum hole in the presence of multiple primary users. 


%==========
%\typeout{}
%\resetdatestamp

\newcommand\Dfrac[2]{\frac{\displaystyle #1}{\displaystyle #2}}
\newcommand{\mathBF}[1]{\mbox{\boldmath $#1$}}
\newcommand{\C}[1]{\mathBF{#1}}

\chapter{Mathematical Layout Styles}

\TeX{} does a marvelous job of setting mathematical formulas, most often
 choosing pleasing spacing.
However, on occasion one should intercede to improve the layout.
This chapter defines a few such occasions.
In addition, this chapter documents some features of the {\tt amsmath}
 package which overcome difficulties in typesetting some mathematical
 forms.
The {\tt amsmath} package is documented 
 in {\it The \LaTeX{} Companion} \cite{Goossens:1997}.

The modified setup is typeset as
\begin{equation}
  G(z) = \begin{cases}
           \Dfrac {P(z)}{1+z^{-1}} & \text{for $p$ even}, \\[1ex]
           P(z)                    & \text{for $p$ odd}.
         \end{cases}
\end{equation}

With the modified definitions, we get the following.
\def\hC#1{\C{\hat{#1}}\vphantom{\C{#1}}}           % hat vector
\def\htC#1{\C{\hat{\tilde{#1}}}\vphantom{\C{#1}}}  % hat, tilde vector
\def\tC#1{\C{\tilde{#1}}\vphantom{\C{#1}}}         % tilde vector
\begin{equation}
\begin{split}
  \C{d}^{(i)} &= \hC{v}^{(i)} - \htC{v}^{(i)} \\
  \C{n}^{(i)} &= \C{u}^{(i)} - \tC{v}^{(i)}
\end{split}
\end{equation}


%==========
%\typeout{}
%\resetdatestamp

\chapter{Tables}

\section{Tables in \LaTeX{}}

Tables of many different sorts can be made with \LaTeX{}.
This chapter gives suggestions on producing tables, along with a number of
 examples.

To illustrate these rules, here is a table and the \LaTeX{} input which was
 used to generate it.
\begin{table}
  \centering
  \Tcaption {Filter specifications}
  \label {T:FSpec}
  \small
  \vskip 1ex
  \renewcommand\arraystretch{1.1}
  \def~{\phantom{0}}
  \def\ExSp#1{\noalign{\vskip #1}}
  \begin{tabular}{cccccc}
    \hline\hline \ExSp{0.4ex}
    Taps  & Transition & Stopband   & Passband & Stop-band & Ultimate \\
    ($N$) & Band       & Weighting  & Ripple   & Rejection & Stop Band \\
          &            & ($\alpha$) & dB       & dB        & dB \\
    \ExSp{0.4ex} \hline \ExSp{0.4ex}
    ~8 &   &   &  0.06~ & 31 & 31 \\
    12 & A & 1 &  0.025 & 48 & 50 \\
    16 &   &   &  0.008 & 60 & 75 \\[1.2ex]
    12 &   &   &  0.04~ & 33 & 36 \\
    16 & B & 1 &  0.02~ & 44 & 48 \\
    24 &   &   &  0.008 & 60 & 78 \\[1.2ex]
    16 &   & 1 &  0.07~ & 30 & 36 \\
    24 &   & 1 &  0.02~ & 44 & 49 \\
    32 & \raisebox{1.5ex}[0pt]{C}
           & 2 &  0.009 & 51 & 60 \\
    48 &   & 2 &  0.006 & 50 & 66 \\[1.2ex]
    24 &   & 1 &  0.1~~ & 30 & 38 \\
    48 & D & 2 &  0.006 & 50 & 66 \\
    64 &   & 5 &  0.002 & 65 & 80 \\[1.2ex]
    48 &   & 2 &  0.07~ & 32 & 46 \\
    64 & \raisebox{1.5ex}[0pt]{E}
           & 5 &  0.025 & 40 & 51 \\
    \ExSp{0.4ex} \hline
  \end{tabular}
  \vskip 1ex
  \begin{tabular}{cc}
    Transition  & Normalized \\
    Code Letter & Transition Band \\
    \ExSp{0.4ex} \hline \ExSp{0.4ex}
    A  & 0.14~~ \\
    B  & 0.10~~ \\
    C  & 0.0625 \\
    D  & 0.043~ \\
    E  & 0.023~ \\
    \ExSp{0.4ex} \hline
  \end{tabular}
  \qquad
  \begin{minipage}[c]{2.1 in}
    \sloppy
    The normalized transition band is the width of the transition band
     normalized to $2\pi$; that is, $(\omega_s - \pi/2) / (2\pi)$.
  \end{minipage}
\end{table}


%========== Appendices
%\appendix

%==========
%\typeout{}
%\resetdatestamp

\chapter{\LaTeX{}Macros}
\label{A:LaTeXmacros}

\section{Numerical Results Overview and Guide }
This section explains how to use the Matlab code that were used to generate the numerical results in this thesis. The capabilities are also discussed. The following table provides an overview of each source code file. All of these files can be found on the attached CD. 

\begin{table}[h]
\begin{tabular}{ll}
\hline
\hline
Source File Name                  & Description                                                                \\ \hline
gaussian\_example.m      & Generate ROC for normal Gaussian situation.              \\
Qian\_Zhang\_Algorithm.m & Implements Qian Zhang Algorithm to achieve MENP parameters.                \\
gaussian\_equal\_var.m   & Generate MROC for Gaussian distributions with equal variances. \\
gaussian\_situ.m         & Generate MROC for normal Gaussian situation.                   \\
JAchisquare.m            & Generate MROC for Chi-Square situation.                        \\
energy.m                 & Generate MROC for energy detection.                            \\
cyclodetection.m         & Generate MROC for cyclostationary detection.                 \\ 
\hline
\end{tabular}
\label{filelist}
\caption{Matlab source files}
\end{table}

To run a program, all of the files in Table \ref{filelist} in the Matlab current work folder. These programs have been successfully run using Matlab 2013b and Matlab 2014a on Linux platform with 12 GB RAM.  When execute JAchisquare.m, gaussiansitu.m or gaussianexample.m, the program loads configurations from init.txt file. When the computation finishes file output.mat stores the result and plotfigure.m execute automatically to generate associated figure. 

The adjustable parameters for ENP (MENP) test are listed in the following table. 


%========== Bibliography
\typeout{}
\begin{singlespace}
  \bibliography{ThesisEx}
  \bibliographystyle{ieeetr}
\end{singlespace}

\end{document}
