\chapter{Conclusions}
In this thesis, we explore the field of spectrum sensing with emphasis on providing different levels of protections for various primary users. The necessary condition for ENP lemma (\rmnum{2}) under certain conditions are provided. We also show that under ENP test the normal direction for a specific point on the ROC surface can be represented by its associated ENP parameters. The relationship between ENP test and Bayesian test are discussed after that. Based on ENP test, we proposed MENP framework which could achieve the largest probability of detection detection under multiple constraints of probability of false alarms under certain conditions. Two methods for acquiring the MENP parameters are provided with simulations. 
After that, we approach MROC surface to represent the relationship between the probability of detection and the constraint conditions.  A sufficient condition for the area achieved by ENP test with positive parameters degenerating to a curve is provided.  Two examples respectively concerning Gaussian distribution and Chi-Square distribution are given to show the properties of M-ROC surface.

We applie the new hypotheses testing frameworking to spectrum sensing in cognitive radio. The energy detector and cyclostationary detector are proposed to detect two primary users. Numerical results are provided for both detectors. It has also been proved that for any false alarm constraints, the enrgy detector can compute the decision rule under MENP framework easily. 
