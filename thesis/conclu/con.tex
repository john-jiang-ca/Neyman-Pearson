\chapter{Conclusions}
In this thesis, we explore the field of spectrum sensing for cognitive wireless communication, with aim of providing different levels of protections for various primary users. The necessary condition for ENP lemma (\rmnum{2}), under certain constraints are provided. We also show that under ENP test the normal direction for a specific point on the ROC surface can be represented by its associated ENP parameters. Furthermore, the relationship between the ENP test and a Bayesian test are discussed. 
A detector based on the ENP test can ensure the largest probability of detecting a free channel under separate constraints of false alarm  for each primary user type. However, with an ENP detector, the achievable false alarm probabilities are limited.  
In this thesis, we proposed the MENP framework which could achieve the largest probability of detection detection under multiple constraints of probability of false alarms under certain conditions. Two methods for finding the MENP parameters are provided. 
After that, we considered the M-ROC surface to represent the relationship between the probability of detection and the constraints. A sufficient condition under the ROC surface of the ENP test with positive parameters degenerates to a curve is provided.  Two examples concerning Gaussian distribution and Chi-Square distribution are given to show the properties of M-ROC surface.

We applied the new hypotheses testing framework to spectrum sensing in cognitive radio. The energy detector and cyclostationary detector are proposed to detect two primary users. Numerical results are provided for both detectors. It has also been proved that for any false alarm constraints, the energy detector can compute the decision rule under MENP framework easily. For cyclostationary detector, there is no direct relationship between the false alarm constraints and the MENP decision rule parameters. In such case, we need to use Qian's algorithm or look-up table method to acquire the MENP decision rule parameters. Both numerical results show that the probability of detection is non-decreasing with respect to the probabilities of false alarm constraints. However, lose the probability of false alarm constraint may not improve the probability of detection. A secure method to increase the probability of detection is checking the system's M-ROC surface and choose a proper probability of false alarm constraints.  

The simulation results for energy detector and cyclostationary detector are provided. 
For energy detector, we use i.i.d. CSCG random variables to simulate the primary signals. For cyclostationary detector, we use real OFDM signals as the primary users' signal.  
In both simulations, we considered the relationship between $P_{f_1}$, $P_{f_2}$, $P_d$, $c_1$, $c_2$ and the MENP parameters.  A comparison between the numerical results and the simulation results are conducted for both detector.
For energy detector, the simulation results conform with numerical results. For cyclostationary detector, even though there are some difference between the mathematical model we build and the OFDM signals we generate, the deviation between the numerical results and the simulation results are negligible.  
