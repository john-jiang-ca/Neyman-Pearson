\section{Relationship to Bayesian Hypotheses Test}
This section considers  the relationship between the ENP Test and the Bayesian Test. 

Consider the following $M+1$ hypotheses concerning an observation $X$
\begin{equation}
\label{equ: 2 pdf}
\begin{split}
H_0:\;\;\;\;&X \sim f_0(x)\\
H_1:\;\;\;\;&X \sim f_1(x)\\
&......\\
H_{M}:\;\;\;\;&X \sim f_M(x)\,,
\end{split}
\end{equation}
where $f_i(x)$ ($i=0, 1, ..., M$) are Probability Density Functions (PDFs). 
In spectrum sensing, $H_0$ denotes the channel is free and $H_i$ ($i = 1, 2, \cdots M$) denotes the channel is occupied by primary user $i$. 
Let $\pi_0, \pi_1, ..., \pi_M$ be the a-prior probabilities of occurrences of hypotheses $H_0$, $H_1$, ..., $H_M$, respectively. 
Based on $x$, a realization of $X$, a Bayesian Test $\delta_B$ is used to decide which hypothesis $x$ associated with.  
Let $C_{ji}$ denote the cost incurred by choosing hypothesis $H_j$ when hypothesis $H_i$ is true. 
Let $\mathcal{C}_j$ ($j=1, ..., M$) denote a subset of $\mathbb{R}$ such that under decision rule $\delta_B$ we choose $H_j$ when $x \in \mathcal{C}_j$. 
Define
\[
P_i(\mathcal{C}_j) = \int_{\mathcal{C}_j} f_i(x)\mathrm{d}x\,.
\]
which is the probabilities that when $H_i$ is correct we choose $H_j$.
In the context of spectrum sensing, the following events will jeopardize the system performance:
\\(1) The system miss a spectrum hole, the cost is represented by $C_{i0}$ ($i = 1, ..., M$). We assume equal such costs, and hence we have $C_{10} = C_{20} = ... C_{M0}$.
\\(2) Primary user $i$ is interfered by secondary user while transferring data, the lost is represented by $C_{0i}$. 

For $C_{ij}$ other than mentioned above, we assume zero. 
From discussion above,  we make following assumptions:
\\(1) $C_{00} = 0$;
\\(2) $C_{ij} = 0$ when $i \neq 0$ and $j \neq 0$;
\\(3) $C_{10} = C_{20} = ... = C_{M0}$.

In the following, we consider the form of $\delta_B$ under such assumptions.
Let $a_0 = C_{10} = C_{20} = ... = C_{M0}$ and $a_i = C_{0i}$ ($i= 1, 2, ..., M$).
The conditional risk for Bayesian Test can be written as 
\begin{equation}
\begin{split}
R_0(\delta_B) &= C_{00}P_0(\mathcal{C}_0) + C_{10}P_0(\mathcal{C}_1) + ... +  C_{M0}P_0(\mathcal{C}_M)\\
&= a_0P_0(\mathcal{C}_1) + a_0P_0(\mathcal{C}_2) + ... + a_0P_0(\mathcal{C}_M)\\
&= a_0 \int_{\bar{\mathcal{C}}_0}f_0(x)\mathrm{d}x\\
&= a_0(1 - \int_{\mathcal{C}_0}f_0(x)\mathrm{d}x)
\end{split}
\end{equation}
\begin{equation}
\begin{split}
R_1(\delta_B) &= C_{01}P_1(\mathcal{C}_0) + C_{11}P_1(\mathcal{C}_1) + ... +  C_{M1}P_1(\mathcal{C}_M)\\  
&= a_1\int_{\mathcal{C}_0}f_1(x)\mathrm{d}x
\end{split}
\end{equation}
\[
......
\]
\begin{equation}
\begin{split}
R_M(\delta_B) &= C_{0M}P_M(\mathcal{C}_0) + C_{1M}P_M(\mathcal{C}_1) + ... +  C_{2M}P_M(\mathcal{C}_2)\\
&= a_M\int_{\mathcal{C}_0}f_{M}(x)\mathrm{d}x\,.
\end{split}
\end{equation}

The total cost function $r(\delta_B)$ can be written as 
\begin{equation}
\begin{split}
\label{r00}
r(\delta_B) &= \pi_0 R_0(\delta) + \pi_1R_1(\delta) + ... +  \pi_MR_M(\delta)\\
&= \pi_0a_0 - \int_{\mathcal{C}_0}\pi_0a_0f_0(x) - \pi_1a_1f_1(x) - ... - \pi_Ma_Mf_M(x) \mathrm{d}x\,, 
\end{split}
\end{equation}
and thus we see that $r(\delta_B)$ is minimized over $\mathcal{C}_0$ if and only if 
\begin{equation}
\label{equ: C}
\mathcal{C}_0 =  \{ x | \pi_0a_0f_0(x) - \pi_1a_1f_1(x) - ... - \pi_Ma_Mf_M(x) \geq 0\}
\end{equation}
on equivalently
\begin{equation}
  \mathcal{C}_0 = \{ x | f_0(x) \geq \sum_{i=1}^{M}\frac{\pi_ia_i}{\pi_0a_0}f_i(x) \}\,.
  \label{2015feb04a1}
\end{equation}

We see that in order to minimize $r(\delta_B)$, $\mathcal{C}_0$ has to be chosen as in \eqref{2015feb04a1}. All the other decision region $\mathcal{C}_i$ ($i = 1, 2, \dots M$) can be chosen arbitrary, since they do not affect \eqref{r00}. A decision rule that employs only $\mathcal{C}_0$ is given by  
\begin{equation}
\label{dec: minimax form}
f_0(x) \substack{H_0 \\ \geq \\ < \\ \bar{H}_0} \sum_{i=1}^{M}\frac{\pi_ia_i}{\pi_0a_0}f_i(x)\,.
\end{equation}

It is seen that \eqref{dec: minimax form} can be used to test $H_0$ against $\bar{H}_0$ only. Assume $k_i = \frac{\pi_ia_i}{\pi_0a_0}$ ($i = 1, 2, \dots, M$),  we have  
\begin{equation}
\label{dec: bay ney}
f_0(x) \substack{H_0 \\ \geq \\ < \\ \bar{H}_0} \sum_{i=1}^{M}k_if_i(x)\,. 
\end{equation}
and it is identical to the ENP decision rule. 
