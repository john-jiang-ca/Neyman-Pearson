% MENP test 

\section{Modified Extended Neyman Pearson Testing}

%define \JUDGEMENT
\def \JUDGEMENT{u(f_0(x) - \sum_{j=1}^{M}k_j f_j(x))}

The following problem needs to be solve in the context of spectrum sensing. 
      \begin{equation}
      \label{equ: problemstate}
      \begin{split}
      \max\;\;\;\;\;\;&P_d\\
      \text{s.t.}\;\;\;\;\;\;&\mathbf{P}_f \leq \mathbf{c}\,,
      \end{split}
      \end{equation}
where $\mathbf{c} \in (0, 1)^M$.

According to ENP Lemma (\rmnum{3}), if a decision rule $\delta^\ast$ satisfies 
\begin{equation}
  \label{2015feb09a0}
  \mathbf{P}_f(\delta^\ast) = \mathbf{c}
\end{equation}
and $\delta^\ast$ can be written in form of 
\begin{equation}
  \label{2015feb09a1}
  f_0(x) \substack{H_0 \\ \geq \\ < \\ \bar{H}_0} \sum_{i=1}^{M}f_i(x)\;\;\;\;(k_i \geq 0)\,,
\end{equation}
then $\delta^\ast$ is the optimal decision rule for \eqref{equ: problemstate}. From \eqref{equ: pf and pd}  we can see 
\begin{equation}
  \label{2015feb09a2}
  P_{f_j} = \int u(f_0(x) - \sum_{i=1}^{M}k_if_i(x))f_j(x)\mathrm{d}x \;\;\;\;(k_i \geq 0)\,.
\end{equation}
Combining \eqref{2015feb09a0} and \eqref{2015feb09a2},
\begin{equation}
\int u(f_0(x) - \sum_{i=1}^{M}k_if_i(x))f_j(x)\mathrm{d}x = c_j\;\;\;\;k_i \geq 0\,.
\label{2015feb09a3}
\end{equation}

From \eqref{2015feb09a3} we can see the problem given in \eqref{equ: problemstate} can be solved by ENP test only when there are parameters $k_i \geq 0$, $i = 1, 2, \cdots, M$ such that  
\begin{equation}
\int u(f_0(x) - \sum_{i=1}^{M}k_if_i(x))f_j(x)\mathrm{d}x = c_j\,.
\label{equ: condition for ENP}
\end{equation}

The case when the given $c_i\;\; (i= 1, 2, ..., M)$ do not satisfy \eqref{equ: condition for ENP} was not considered so far. 


Next we present the Modified Extended Neyman Pearson Test (MENP) for solving \eqref{equ: problemstate} for all $\mathbf{c}$.
Before we proceed, we will define some notations for easy presentation.
Define the set 
\[\mathcal{A}_\mathbf{c} = \{
  \mathbf{P}_f | 0 \leq P_{f_i} \leq c_i
  \;\;i=1,  ..., M\}\,.
\]
and the set 
\[
\alpha^+ \triangleq \{\mathbf{P}_f | P_{f_i} = \int_{-\infty}^{\infty} \JUDGEMENT f_i(x) \mathrm{d}x, \text{where\;\;} k_i \geq 0 \;\;i=1, ..., M\}\,.
\]
As it is defined in section 2.2.2, $F(\mathbf{a})$ denotes the largest $P_d$ that can be achieved while keeping $\mathbf{P}_{f} = \mathbf{a}$.  

\noindent \textbf{Modified Extended Neyman Pearson Test}
\noindent \textit{
  \\\textnormal{(\rmnum{1})} Assume $\mathbf{c} \in \alpha^+$. Then there must be a $\mathbf{k}^0 = [k_1^0, k_2^0, ..., k_M^0]^T$ with $k_i^0 \geq 0\;\;i=1, ..., M$ satisfying
}
\begin{equation}
\label{equ:Pf}
  P_{f_i}^0 = \int_{-\infty}^{\infty} u(f_0(x) - \sum_{j=1}^{M}k_j^0f_j(x))f_i(x)\mathrm{d}x = c_i \;\; (i= 1, 2, ..., M)\,.
\end{equation}
\textit{
    and the decision rule $\delta^0$ is:
}
\begin{equation}
\label{equ:decision rule}
\delta^0:\;\;\;\;f_0(x) \substack{H_0 \\ \geq \\ < \\ \bar{H}_0} \sum_{i=1}^{M}k_i^0f_i(x)
\end{equation}
\textit{
  \noindent \textnormal{(\rmnum{2})} Assume $\mathbf{c} \notin \alpha^+$ and $\mathbf{c} \in (0, 1)^M$. Let $\mathcal{C} = \mathcal{A}_{\mathbf{c}} \cap \alpha^+$, and $\mathbf{a}^0 = [a_1^0, a_2^0, ..., a_M^0]^T \in \mathcal{C}$ be such that
}
\begin{equation}
\label{equ: F0}
\max_{\mathbf{a} \in \mathcal{C}}\;\;\;\;F(\mathbf{a}) = F(\mathbf{a}^0)\\
\end{equation}
\textit{
Since $\mathbf{a}^0 \in \mathcal{A}_{\mathbf{c}} \cap \alpha^+$, from \textnormal{(\rmnum{1})} we have that there exists a vector $\mathbf{k}^0$ such that \eqref{equ:decision rule}  maximizes $P_d$ under the constraints $P_{f_i} = a_i^0, \;i=1, ..., M$. Hence, since $a_i^0 \leq c_i^0, \;i=1, ..., M$, this decision rule solves \eqref{equ: problemstate}.
}

Comparing the ENP test and the MENP test, we can see there is repetition between them. When $\mathbf{c} \in   \alpha^+$, the ENP test and the MENP test provide the same solution. However, in the situation when $\mathbf{c} \notin \alpha^+$, $\mathbf{c} \in (0, 1)^M$, only the MENP test can provide the solution for \eqref{equ: problemstate}. 

Next we will show that the \textbf{Modified Extended Neyman Pearson Test} can provide an optimal solution for \eqref{equ: problemstate}.

\textbf{PROOF}
MENP (\rmnum{1}) is a direct conclusion from ENP Lemma (\rmnum{3}). In the following,  we give the proof for MENP (\rmnum{2}). First we will show $\mathcal{C}$ is not an empty set. 
Assume $\delta^\ast$ is the optimal decision rule for \eqref{equ: problemstate} (According to \textbf{Lemma 1}, such decision rule always exists), we have  
\begin{equation}
\label{condition 1}
P_{f_i}(\delta^\ast) \leq c_i\;\;\;\;i=1,  ..., M\,.
\end{equation}
From \textbf{Lemma 2}, there exists non-negative constants $k_1^\ast, ..., k_M^\ast$ such that $\delta^\ast$ can be written in form of 
\begin{equation}
f_0(x) \substack{H_0 \\ \geq \\ < \\ \bar{H_0}} \sum_{i=1}^{M}k_i^\ast f_i(x)\,.
\label{2015feb20a1}
\end{equation}
and thus $P_d(\delta^\ast)$ and $P_{f_i}(\delta^\ast)$ can be written as
\begin{equation}
\begin{cases}
\label{TEMP10}
P_{d}(\delta^\ast) &= \int_{-\infty}^{\infty} u(f_0(x) - \sum_{j=1}^{M}k_j^\ast f_j(x)) f_0(x) \mathrm{d}x\\
P_{f_i}(\delta^\ast) &= \int_{-\infty}^{\infty} u(f_0(x) - \sum_{j=1}^{M}k_j^\ast f_j(x)) f_i(x) \mathrm{d}x
\end{cases}
\end{equation}
where $k_i^\ast \geq 0$.

From the definition of $\mathcal{A}_c$ and \eqref{condition 1}, we know $\mathbf{P}_f(\delta^\ast) \in \mathcal{A}_c$. 
From the definition of $\alpha^+$ and \eqref{TEMP10}, we know $\mathbf{P}_f(\delta^\ast) \in \alpha^+$. 
Thus $\mathbf{P}_f(\delta^\ast) \in \mathcal{A}_\mathbf{c} \cap \alpha^+$, i.e. $\mathbf{P}_f(\delta^\ast) \in \mathcal{C}$. Since  $\mathbf{P}_f(\delta^\ast) \in \mathcal{C}$, we know set $\mathcal{C}$ is not empty.
Assume $\mathbf{a}^0$ satisfy
\begin{equation}
\label{a0}
\max_{\mathbf{a}\in\mathcal{C}} F(\mathbf{a}) = F(\mathbf{a^0})
\end{equation}
and decision rule $\delta^0$ maximizes $P_d$ under the constraints $P_{f_i} = a^0_i$ ($i = 1, ..., M$), where $\mathbf{a}^0 \in \alpha^+$. From MENP (\rmnum{1}) we know there exists non-negative $k_1^0, ..., k_M^0$ such that $\delta^0$ can be written in form of 
\[
\delta^0:\;\;\;\;f_0(x) \substack{H_0 \\ \geq \\ < \\ \bar{H_0}} \sum_{i=1}^{M}k_i^0f_i(x)\,.
\] 
From the definition of $F(\mathbf{a})$, we know $F(\mathbf{a^0}) = P_d(\delta^0)$.

In the following, we  will prove $\delta^0$ is the optimal solution for \eqref{equ: problemstate}.
To prove $\delta^0$ is the optimal solution for \eqref{equ: problemstate}, we only need to show $P_d(\delta^0) = P_d(\delta^\ast)$. This is because for any decision rule $\delta'$ whose associated $\mathbf{P}_f \leq \mathbf{c}$, we have $P_d(\delta^\ast) \geq P_d(\delta')$. If $P_d(\delta^0) = P_d(\delta^\ast)$, then we can conclude $\delta^0$ achieves the largest $P_d$ among all decision rules whose associated $\mathbf{P}_f$ is no larger than $\mathbf{c}$, i.e. $\delta^0$ is the optimal solution for \eqref{equ: problemstate}. Next, we will show $P_d(\delta^0) = P_d(\delta^\ast)$.   

Assume  $\delta^0$ does not achieve the largest $P_d$ under constraints $P_{f_i} \leq c_i$ ($i=1, ..., M$). Then  we must have 
\begin{equation}
\label{condition 2}
P_{d}(\delta^\ast) > P_d(\delta^0)\,.
\end{equation}

Let $\mathbf{a}' = \mathbf{P}_f(\delta^\ast)$. 
Since $\delta^\ast$ can be written in form of \eqref{2015feb20a1}, from ENP Lemma (\rmnum{2}), we know decision rule $\delta^\ast$ achieves the largest $P_d$ while keeping $\mathbf{P}_f = \mathbf{a}'$, i.e.  
$F(\mathbf{a}') = P_d(\delta^\ast)$. 

Since $P_d(\delta^\ast) > P_d(\delta^0)$, $F(\mathbf{a}^0) = P_d(\delta^0)$ and  $F(\mathbf{a}') = P_d(\delta^\ast)$, we have $F(\mathbf{a'}) > F(\mathbf{a^0})$. 

As $\mathbf{a}' \in \mathcal{C}$ and $F(\mathbf{a'}) > F(\mathbf{a^0})$, it is contradictory with the condition that $\mathbf{a^0}$ satisfy \eqref{a0}.

Hence we conclude $\delta^0$ is the optimal solution for \eqref{equ: problemstate}. $\blacksquare$

Both MENP (\rmnum{1}) and MNEP (\rmnum{2}) show that the optimal decision rule for \eqref{equ: problemstate} is an ENP decision rule  with non-negative parameters such that: (a) its $P_{f_i}$ is smaller or equal to its associated constraints; (b) it maximizes $P_d$ among all ENP decision rule with non-negative parameters satisfying (a).

Base on above discussion, 
next we present a look-up table method of finding $\mathbf{k}$. For a specific $\mathbf{c}$, we conduct an exhaustive search over a grid of $k_1, ..., k_M$ ($k_i \geq 0$) depending on the required accuracy and store these results ($P_d, \mathbf{P}_f, \mathbf{k}$) into a lookup table $T_1$. 
%Then we check if there exists a item in table $T_1$ satisfying $\mathbf{P}_f = \mathbf{c}$. If so, get the item's associated $\mathbf{k}$ as the MENP parameters. If not,  
Then we check each item in Table $T_1$ (each item includes the information of $\mathbf{k}$ and its associated $P_d$ and $\mathbf{P}_f$) and put the item into table $T_2$ if the item satisfies: $\mathbf{P}_f \leq \mathbf{c}$. At last, we check each item in $T_2$ and find the item whose $P_d$ is the largest among all items in Table $T_2$. Since this item contains the MENP parameters to achieve this $P_d$, we can get the corresponding $\mathbf{k}$. 


