\chapter{Introduction}
Future wireless communication systems will require higher transmission speed in order to deliver large amount of data \cite{pelcat20133gpp}. The main obstacle for providing high speed wireless communication services is the limitation of available radio frequency resource. In order to make potential radio system use spectrum more efficiently than in the past, advanced technologies are created, such as Cognitive Radio (CR) \cite{nonotice}. 

CR is an intelligent wireless communication system which is able to monitor its operating spectrum environments and change its transmitter parameters accordingly to best match these conditions \cite{wang2011advances, a001}. It provides a novel approach for the coexistence of different wireless communication systems in the same frequency band. In CR communication system, users are divided into two types: primary user and secondary user. A primary user is legacy user or a licensed user who has priority on the particular part of frequency band. The CR system must ensure highly reliable communications whenever and wherever needed by primary users \cite{a001}. Potential primary users may include cellphone communication systems (e.g., GSM, CDMA or LTE), Satellite TV Broadcast systems, microphone communication systems and so on. A secondary user or a cognitive user is refer to someone who has lower right to access the particular spectrum. A secondary user can utilize the spectrum only when the particular frequency is idle but has to vacate the frequency band as soon as any primary user becomes active. 
In order to real-time monitor the utilization of the spectrum, the spectrum sensing component is introduced into CR communication system \cite{buddhikot2007understanding, tandra2009spectrum}.   

Spectrum sensing is the task of identifying vacant band under certain radio frequency environment and detecting the primary users with high probability of detection  \cite{umar2012spectrum}. 
%A spectrum sensing system includes at least one measuring device and at least one testing device. The role of measuring device is transferring the RF stimuli into a suitable test statistics. The role of testing device is decide the spectrum state according to the output(s) of measuring device(s).
According to the number of cognitive device, spectrum sensing can be divided into single-user sensing (local detection) and cooperative sensing \cite{wang2011advances, akyildiz2011cooperative, ma2008soft, axell2010overview}. In single-user sensing, the spectrum sensing system makes the decision about the active or inactive of incumbent based on the observation of one cognitive device. The main advantage of single user sensing is its low hardware complexity. However, it has been shown that its performance decreases heavily in challenging propagation environments (e.g. multipath fading channel, shadowing). In such scenario, cooperative sensing is preferred. Cooperative sensing is a detection methods where multiple sensing device cooperate to decide the radio frequency's state \cite{ganesan2005cooperative, arslan2007cognitive}. While cooperative sensing improving detection performance in challenging environments, it requires a more complicated system design and brings more sensing time, energy consumption \cite{akyildiz2011cooperative}.  
Moreover, the communication between sensing devices can lead to extra spectrum occupation. 

Various sensing methods have been approached for identifying the spectrum usage opportunities. Main sensing techniques include  energy detection, cyclostationary detection and waveform detection. 

Energy based detector is the simplest spectrum sensing detector and has been extensively used in radiometer \cite{cabric2004implementation, poor1994introduction, urkowitz1967energy}. It uses the energy of received signal to identify the channel status.Since it does not require complicated digital signal processing, it has low energy consumption and low hardware complexity.  Another favorable aspect of energy detector is the detection process does not rely on  any previous information about primary user's signal.  
The shortcuts of energy detector includes: (1) it takes a prohibitory long time to detect primary user if the SNR is low; (2) it requires the information about the variance of noise, which in some case is hard to estimate; (3) its performance is not satisfying in multipath channel; (4) it is unable to detect spread spectrum signals like CDMA \cite{urkowitz1967energy, akyildiz2011cooperative}. 
In \cite{akyildiz2011cooperative} the author shows that the first three problems can be mitigated by diversity gain result from cooperation, which make it a suitable detection schemes in cooperative sensing for non-CDMA signals.  

Cyclostationarity feature based detector identifies the spectrum status by exploiting autocorrelation properties of the received signals, which is brought by the periodicity structure in the signal \cite{goldsmith2009breaking}. Common used periodicity includes sinusoidal carriers,  regularly transmitted pulse trains, pilot sequence and cyclic prefixes (CP) in OFDM signals \cite{akyildiz2011cooperative, umar2013comparative}.  
Comparing to energy detection, cyclostationary based detector is more robust to noise uncertainty and has better performance in low SNR regime \cite{umar2013comparative}. However since the cyclostationarity detector requires high timing accuracy when sampling, the hardware is more complex than energy detector \cite{yucek2009survey}. Another disadvantage of cyclostationarity based detector is its performance under fading channel is not satisfying \cite{tandra2007snr}.   

Waveform based detector identifies the channel status by exploiting the known pattern in primary user's signal, such as pilot.The favourable aspect of this detector lies in it can achieve a high probability of detection is a comparatively short time and is able to achieve a good performance in a shadowing environment \cite{tang2005some}. However, since the detector need to extract the pilot sequence, it relies on dedicated synchronization and accurate information about the primary user's transition parameters.  

%Matched filter based detector has the similar structure as a primary user receiver. Even in low SNR environment, it can still achieve a relatively high probability of detection in a short detection time \cite{tandra2005fundamental}. However since it needs to demodulate the received signal, it requires perfect information of primary signal \cite{yucek2009survey}. Another unfavorable aspect of this detector is its high power consumption, which results from its complicated hardware structure.  

Resent years, some new sensing algorithms and digital signal processing techniques have been introduced for specific cognitive sensing situation. Among them is wavelets based detection \cite{tian2007compressed, sun2013wideband, sun2013wideband2}. In wavelets based detection, sub-Neyquist sampling and compressed sensing are used to estimate the power spectrum density in a wide band. After that it exploits wavelets to detect the PSD edges, which correspond to transition from a occupied status to a idle status, and to estimate the power between each edges. In this way, the wideband spectrum utilization is known and the secondary users can exploit one or several idle sub-bands to communicate. 

In order to detect a free frequency channel, a spectrum sensing scheme solves a binary hypotheses testing problem, where the null hypothesis refers to the event that a user is not using the channel and the alternative hypothesis refers to the event that a user is occupying the channel. If the a-prior probabilities of null and alternative hypotheses are known, then such a problem can be solved using a Bayesian framework \cite{poor1994introduction}. The situation when Baysian framework could be used is considered in \cite{zeng2010review} and the performance of such scheme is analyzed.
Even though the Baysian framework can be used in some situations, in more often cases, the a-prior probabilities are not available. In such situations Neyman Pearson testing is employed \cite{poor1994introduction}.When there is a single type of primary user (two hypotheses), NP testing will achieve the largest probability of detecting a vacant channel when no primary user is present under a constraint on the probability that the channel will be declared vacant when in fact a primary user is present.
Traditional NP testing has been extensively used for two hypotheses problem. In such cases,  the performance is characterized by the Receiver Operating Characteristic (ROC) curve, which represents the relationship between probability of detection and probability of false alarm \cite{poor1994introduction}. 

\section{Thesis Objectives}
In spectrum sensing application, there may be more than one type of primary users. For example in IEEE 802.22 \cite{shellhammer2008spectrum} the channel can be occupied by Analog-TV Digital-TV and wireless microphone. 
In such cases, an extended NP (ENP) test can provide important advantages \cite{zhang1999design}. A detector based on ENP testing can ensure the largest probability of detection under a separate constraint of false alarm for each primary user type \cite{LehmannTest}.
With an ENP detector, the achievable false alarm probabilities are limited. This motivates us to analyze the properties of the ENP test and propose the Modified Extended Neyman Pearson (MENP) test, that circumvents the limitations of the false alarm probabilities. 

Performance analyses are setup to illustrate the performance of our approached spectrum detectors for OFDM systems. 

\section{Thesis Organization and Contribution}
The Thesis is organized as follows. We review ENP test framework and proposed MENP test framework in chapter 2. In particular, we revist the Lemma of Extended Neyman Person and show an example to implement ENP test. After that we illustrate two lemmas concerning the ROC surface of ENP test. After that, a necessary condition of ENP test is given with proof. Once the properties are described, we approached the MENP test, which focus on acquiring the largest probability of detection under any probability of false alarm constraints. At the end, an algorithm to achieve ENP parameters is reviewed and applied to MENP framework.

In Chapter 3, M-ROC, the figure depicting the relationship between probability of detection and probability of false alarm constraints, is presented.  We then show two examples of M-ROC surface respectively concerning the Gaussian situation and Chi-Square situation. Finally we discuss the shape of M-ROC surface.

Once the MENP framework is presented, we apply MENP framework to energy based detector and cyclostationary based detector in Chapter 4. We look into the performance of both detector for OFDM signals under AWGN channel. 

Chapter 5 provides a recapitulation of this thesis and recommends some directions for future work.  
