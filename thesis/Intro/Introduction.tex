\chapter{Introduction}
Future wireless communication systems will require higher transmission speed in order to deliver large amount of data \cite{pelcat20133gpp}. The main obstacle for providing high speed wireless communication services is the limitation of available radio frequency resources. In order to make radio systems use the radio spectrum more efficiently than in the past, advanced technologies, such as Cognitive Radio (CR) \cite{federal2005notice}, have been developed.  

A CR is an intelligent wireless communication system which is able to monitor its operating spectrum environments and change its transmitter parameters to best match spectrum conditions \cite{wang2011advances, a001}. This technology provides a novel approach for the coexistence of different wireless communication systems in the same frequency band. In a CR communication system, users are divided into two types: primary users and secondary users. A primary user is a legacy user, or a licensed user, who has priority on the particular frequency band. The CR system must ensure highly reliable communications whenever and wherever needed by primary users \cite{a001}. Potential primary users may include cellphone communication systems (e.g., GSM, CDMA or LTE), Satellite TV Broadcast systems, microphone communication systems etc.. A secondary user, or a cognitive user, is referred to someone who has lower priority to access the particular frequency band. A secondary user can utilize the spectrum only when a particular channel is idle, and has to vacate it as soon as any primary user becomes active. 
In order to monitor the utilization of the spectrum in real-time, a spectrum sensing functionality must be introduced into a CR communication system \cite{buddhikot2007understanding, tandra2009spectrum}.   

Spectrum sensing refers to the task of identifying vacant frequency channels in a certain radio frequency band by detecting primary users with high probability \cite{umar2012spectrum}. 
%A spectrum sensing system includes at least one measuring device and at least one testing device. The role of measuring device is transferring the RF stimuli into a suitable test statistics. The role of testing device is decide the spectrum state according to the output(s) of measuring device(s).
Spectrum sensing systems can be divided into single-user sensing (local detection) and cooperative sensing \cite{wang2011advances, akyildiz2011cooperative, ma2008soft, axell2010overview}. In single-user sensing, the system makes a decision about the status of a frequency channel based on the observation from one sensing device. The main advantage of single user sensing is its low hardware complexity. However, it has been shown that its performance deteriorates significantly in challenging propagation environments (e.g. multipath fading, shadowing) \cite{akyildiz2011cooperative}. In such scenarios, cooperative sensing is preferred. Cooperative sensing employs  multiple sensing devices that cooperate to detect the status of a frequency channel \cite{ganesan2005cooperative, arslan2007cognitive}. While cooperative sensing improves detection performance in challenging environments, it requires a more complicated system design,  longer sensing time,  and increased energy consumption \cite{akyildiz2011cooperative}.  
Moreover, the communication between sensing devices can lead to extra spectrum usage. 
Various sensing methods have been considered for identifying spectrum usage opportunities. Main sensing techniques include  energy detection, cyclostationary detection and waveform detection. 

Energy based detection is the simplest spectrum sensing techniques, that has been extensively used in radiometers \cite{cabric2004implementation, poor1994introduction, urkowitz1967energy}. It employs the energy of the received signal to identify the channel status. Since it does not require complicated digital signal processing, it has low energy consumption and low hardware complexity.  Another favorable aspect of an energy detector is that it does not require  any information about the primary user's signal.  
The shortcomings of an energy detector include: (1) long time to detect a  primary user if the SNR is low; (2) requires information about noise variance, which in some case is hard to estimate; (3) performance is not satisfying in multipath channels \cite{akyildiz2011cooperative}; (4) unable to detect spread spectrum signals like CDMA \cite{urkowitz1967energy, akyildiz2011cooperative}. 
In \cite{akyildiz2011cooperative} the authors show that the first three problems can be mitigated by diversity gain from cooperation,  making it a suitable techniques in cooperative sensing for non-CDMA signals.  

Cyclostationarity feature based detector identifies the channel status by exploiting autocorrelation cyclic properties of received signals, that are a byproduct of periodicity features \cite{goldsmith2009breaking}. Such periodicity features are induced by  sinusoidal carriers,  regularly transmitted pulse trains, pilot sequences and cyclic prefixes (CP) in OFDM signals \cite{akyildiz2011cooperative, umar2013comparative}.  
When compared with an energy detection, a cyclostationary based detector is more robust to noise uncertainty and has better performance in the low SNR regime \cite{umar2013comparative}. However since a cyclostationarity detector requires high timing accuracy when sampling, the hardware is more complex than that of an energy detector \cite{yucek2009survey}. Another disadvantage of cyclostationarity based detector is that its performance over fading channels is not satisfactory\cite{tandra2007snr}.   

Waveform based detection identifies the channel status by exploiting known patterns in the primary user's signal, such as pilot sequences. The favorable aspect of such a detector is that it can achieve a high probability of detection in a comparatively short time and is able to provide  good performance in a shadowing environment \cite{tang2005some}. However, since the detector needs to extract the pilot sequence, it relies on dedicated synchronization and accurate information about the primary user's signal structure.  

%Matched filter based detector has the similar structure as a primary user receiver. Even in low SNR environment, it can still achieve a relatively high probability of detection in a short detection time \cite{tandra2005fundamental}. However since it needs to demodulate the received signal, it requires perfect information of primary signal \cite{yucek2009survey}. Another unfavorable aspect of this detector is its high power consumption, which results from its complicated hardware structure.  

%Recent years, new sensing algorithms and digital signal processing techniques have been introduced for specific spectrum sensing situations. One of them is wavelets based detection \cite{tian2007compressed, sun2013wideband, sun2013wideband2}. In wavelets based detection, sub-Neyquist sampling and compressed sensing are used to estimate the power spectrum density in a wide band channel. After that it exploits wavelets to detect the power spectrum density (PSD) edges, which correspond to transition from a occupied status to a idle status, and to estimate the power between each edges. In this way, the wideband spectrum utilization is known and the secondary users can exploit one or several idle sub-bands to communicate. 

In order to detect a free frequency channel, a spectrum sensing scheme solves a binary hypotheses testing problem, where one hypothesis refers to the event that a user is not using the channel and the other hypothesis refers to the event that a user is occupying the channel. If the a-prior probabilities of these hypotheses are known, then such a problem can be solved by using a Bayesian framework \cite{poor1994introduction}. The situation when a Baysian framework could be used is considered in \cite{zeng2010review}, where  the performance of such scheme is analyzed.
Even though the Baysian framework can be used in some situations, in more often cases, the a-prior probabilities are not available. In such situations Neyman Pearson testing is employed \cite{poor1994introduction}.When there is a single type of primary user (two hypotheses), NP testing will achieve the largest probability of detecting a vacant channel when no primary user is present under a constraint on the probability that the channel will be declared vacant when in fact a primary user is present.
Traditional NP testing has been extensively used for two hypotheses problems. In such cases,  the performance is characterized by the Receiver Operating Characteristic (ROC) curve, which represents the relationship between the probability of detection and the probability of false alarm \cite{poor1994introduction}. 

\section{Thesis Objectives}
In spectrum sensing applications, there may be more than one type of primary users. For example in IEEE 802.22  the channel can be occupied by Analog-TV Digital-TV and wireless microphone\cite{shellhammer2008spectrum}. 
In such cases, an extended NP (ENP) test  \cite{LehmannTest} can provide important advantages. A detector based on the ENP testing can ensure the largest probability of detecting a free channel under separate constraints of false alarm for each primary user type \cite{LehmannTest}.
However, with an ENP detector, the achievable false alarm probabilities are limited. This motivates us to analyze the properties of the ENP test and propose the Modified Extended Neyman Pearson (MENP) test, that circumvents the limitations of the false alarm probabilities. 
Furthermore, we apply the MENP test to energy and cyclostationary based spectrum detector and investigate the performance for both detectors in the presence of two primary users.

\section{Thesis Organization and Contribution}
This thesis is organized as follows. We review the ENP framework and propose the MENP framework in Chapter 2. In particular,  we consider  a necessary condition proof for an  ENP test followed by  two lemmas concerning the ROC surface of the ENP tests. 
We then present the MENP framework aiming to  achieve the largest probability of detection  under any probability of false alarm constraints.
Then an algorithm which provides a way of calculating the MENP parameters is considered. 

In Chapter 3, the Modified Receiver Operating Characteristic (M-ROC), the relationship between the probability of detection and probability of false alarm constraints, is considered.  We then present two examples of M-ROC surface one associated with a Gaussian problem and the other with a Chi-Square problem. 

Once the MENP framework is presented, we apply it to energy based detection and cyclostationary based detection in Chapter 4. We consider  performances of both detectors for the situation when there are two primary users.  

Chapter 5 provides a summary of this thesis, and recommends some directions for future work.  

The contributions of this thesis are:

$\bullet$ Proposed the ENP testing framework for spectrum sensing and analyze its properties.  

$\bullet$ Proposed a novel detection scheme,  the MENP test, which generalizes the ENP  test to  achieve the largest probability of detection under any probability of false alarm constraints. 

$\bullet$ Provided numerical results to show the performance of the MENP test under Gaussian and Chi-Square situations. 

$\bullet$ Using the MENP test, energy and cyclostationary based detectors for spectrum sensing in the presence of multiple primary users. 
